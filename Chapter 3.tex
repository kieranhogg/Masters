\section{Chapter 3}
4,000 - 5,000 words
\subsection{Methodology}

\subsubsection{The purpose of the research}
The aim of the research is to investigate innovation within schools in the UAE. There will be a focus on the understanding of innovation itself, something that has come out of the literature review and preliminary research. An additional areas of focus is the reaction, if any, from schools to the introduction of innovation as an inspection. Finally, some schools, if applicable, will be chosen to look at any innovative practice occurring. The aim being to produce a summary or reduction of best practice for innovation in schools.

\subsubsection{The nature of the knowledge which is created}

\subsubsection{Types of research}

As there are multiple areas to look at within the research, it is obviously important to look at different types of research methods to ascertain the best method for data collection in each instance. In the first instance, the seminal \textit{Research Methods in Education} by \citet{Cohen2005} is the starting point to identify possible research methods that might be suitable for each task.

In simple terms, the research aims to study the effect of an external influence (the innovative drive) on schools. The research methodology options are therefore already narrowed; the change has already happened and therefore does not need to be discovered or inferred. This strongly rules out the \textit{ex post facto} methodology as the cause is already known to us.

Naturalistic enquiry was included as a possible research method, as the "intention of the research is to create as vivid a reconstruction as possible of the culture or groups being studied", \cite{Cohen2005} which is not not only a possibility, but a good method of finding out the current impact of the changes. Elements of a naturalistic enquiry which seem to suggest it would be a good fit include: "researchers do not know in advance what they will see or what they will look for", "research designs emerge over time", "social research should be conducted in natual, uncontrived, real world settings with as little intrusiveness as possible by the researcher", and "data are analysed inductively, with constructs deriving from the data during the research". The idea of inductive, or iterative investigation is one that is very close the anticipated model as no-one knows what changes have happened and how people are perceiving them. 

The concern with naturalistic enquiry would be the lack of structure; while it seems appropriate, there is a danger that if the research designs emerging were weak or likewise for the data, the inductive approach might break down and leave weak research.

Surveys were another possible approach as it "set[s] out to describe and to interpret \textit{what is}." \cite[p. 169]{Cohen2005} As the impact of the changes are unknown, describing the current state of those changes would be a reasonable method. The interpretation was a slight concern given the (anticipated) relatively small sample size and the difficulty in having a position of strength in order to interpret or draw any strong conclusions from the data.

Action research and experiments were deemed to be too narrow and practical; they would have to focus on something such as the effect of innovative teaching on lessons or students whereas it was decided to look at the whole school and community impact.

Given the different areas of research, the decision was made to take a pragmatic approach to research which will encompass naturalistic enquiry for discovery of the influence within schools and school culture, and surveys to find out the numbers behind the changes.  A pragmatic approach fits well with this attempt to find the holistic effect of innovation, it "[v]iews current truth, meaning, and knowledge as tentative and as changing over time",  \cite[p. 18]{Johnson_2004} "[e]ndorses eclecticism and pluralism (e.g., different, even conflicting, theories and perspectives can be useful; observation, experience, and experiments are all useful ways to gain an understanding of people and the world)"  and "[e]ndorses eclecticism and pluralism (e.g., different, even conflicting, theories and perspectives can be useful; observation, experience, and experiments are all useful ways to gain an understanding of people and the world)".

\subsubsection{The reason for your choice of research methodology}



\subsubsection{The possible measures for trusting the research (validity and reliability)}
\subsubsection{The nature and form of what may be generalised from the research.}

While the difficulty in getting accuracy in responses due the differing opinions on what innovation in education is, this will not diminish the validity of those responses, in themselves they are valuable data points. The generalisation of the research will be twofold: to what extent is innovation beneficial in schools and how best to achieve an innovative school; and reaction to a community-wide initiative and the level of success in doing so.

\subsubsection{The power relationships and involvement of stakeholders and the effects that this may have.}

It is important to note of the power dynamic in this ecosystem, one which is crucial to bear in mind when conducting the research, or indeed reading any results or conclusions. The innovation initiative is one that has been proposed and distributed by an external body to schools, a body which schools are otherwise held account to. Anyone participating in this research, especially anyone in a more senior leadership role, will therefore have a tendency to portray an image that perhaps be more generous that may actually be the case. To minimise this effect, the research will be conducted face-to-face where it is thought that a disclaimer or explanation would facilitate more honest dialogue. Additionally, all quantative research conducted will be anonymous and where possible, independent of institution in order to gain the most accurate data feasible.

\subsubsection{The ethical basis of the research.}
Ethically, the sole issue is anonymity. The relative level of success with the initiative will depends on a large number of factors, some which may not reflect positively on the staff or schools taking part. Individuals taking part have been guaranteed their anonymity as well as making the schools taking part anonymous. With a range of schools and the nature of innovations such as this, there will inevitably be some results which look less favourable for some schools. It is also important to ground this research in one very specific area which is a small part of inspection framework which means no inferences can be made on the schools in the wider context.

This section should also:
•	Identify any bias that the researcher may bring to the research and how they intend to overcome this
•	Identify any variables and how they can will be dealt with
•	Outline how your research outcomes could be generalisable
•	The Tests for Truthfulness, Reliability and Validity that may be applied
•	The Involvement and Relationships between Researcher, and Subjects

\subsection{Methods of Data Collection}
Throughout almost every area of the data collection, one of the key concerns was teachers, unlike many professions, do not have "free time" during the day. That is, during contact hours, a teacher could not reasonably be expected to answer a phone or reply to an email. Additionally, outside of contact hours, many teachers use that time to plan and assess student work. Teachers' time is therefore relatively costly, and asking them to take time out to provide data altruistically is a huge challenge.

The first area of research is finding out data from teachers. This encompasses looking at definitions and examples of innovation, as well as seeing if they believe their school has changed in response to the changes. As we are collecting data about the current situation, it had to be decided to choose from interviews, questionnaires, accounts, observations and personal constructs as possible methods, as taken from \citet{Cohen2005}.
The most constraining element in this area of data collection is the time (and potentially distance) needed to provide any face-to-face data collection for all but questionnaires. Needing to contact as many different people in different schools as possible meant interviews were time-prohibitive, even if they were demoted from in-person to over the phone. Accounts, observations and personal constructs, all required either a prohibitively large period of time from me, or to push the responsibility onto the teachers themselves to produce the data, both not acceptable. By reason of elimination, questionnaires were the only method where many teachers' opinions could be solicited (anticipating a level of non-completion) and the time spent both all parties would be minimised.

While the data required from teachers is mostly quantitative, the more open-ended questions should be fairly 
TODO: look at below, does it fit?

There will be three main methods of data collection for the research.

\subsubsection{Questionnaires}
Questionnaires will be used to collect the main part of the data due to the nature of data required. The questions of how well understood innovation is, and to what extent it had been implemented by schools are ones that could be answered by teachers. Ideally, interviews would have been conducted in order to establish meaning, inference and to gain clarity. Due to the time constraints mentioned earlier however, questionnaires were chosen as a method which could be easily distributed which would mean the anticipated low interest rate could be mitigated as much as possible with a larger sample size which is simply not feasible with an interview.

The questionnaire will be delivered digitally via Google Forms for the ease of distribution, completion and analysis, and structured questionnaire will be used. There will be a range of information that will need to be solicited from the teachers, so a mixed questionnaire consisting of both closed and open ended questions will be used. There will need to be some simple closed questions which can be measured quantitatively such as whether their school has made any changes regarding innovation and whether they consider their school innovative. The latter question will not be used exclusively, but will form a part of the pre-selection of schools to examine more closely as part of observational case studies.

There will be some questions which will required teachers to rate things on a scale. In particular, what they think the impact of innovation is, and to rate their school's level of innovation in chosen areas. The former is to determine whether innovation is something that is actually useful, up until this point there has almost been an assumption that it is a always a positive thing. The latter is to determine if there a prevailing 'type' of innovation that is more common than others in schools.

Respondents will be confidential but not anonymous. As there is a plan to select certain schools for further study, respondents' schools will be required in order to follow up, however their name is not required. 

While the aim of the questionnaires is to give a picture of the UAE that is as accurate as possible, the nature of the research is that it is not required for a statistically significant sample for the data to be usable. If there are only a handful of schools taking part, the data provided on those is sufficient to provide a narrative and discussion of what has taken place in those schools. If the research was to, for example, investigate a link between innovation and improvement in student performance then this would obviously not be the case. This will be seen further when investigating individual schools; the aim is not to draw conclusions about every school in the UAE, just to investigate what is taking place in some schools.
Despite that, the questionnaire target will be around 100 teachers. At a conservative estimate of five teachers in a school return rate, therefore 20 schools are required and 30 schools will be contacted at random.

\subsubsection{Interviews}
1.	How many interviews will you decide to conduct?
2.	Who will you want to interview and why?
3.	How will you decide whom to interview?
4.	What kind of interview will you conduct - e.g. a formal interview with set questions, a more informal interview with some questions but space for the interviewee to give you a more personal reflection or a completely unstructured interview around a given topic
5.	How will you record your interviews?
6.	How will you need to transcribe the interviews?
7.	How you will interpret your interviews

As identified by Sharma et al., a portion of their framework relates to management and leadership. Interviews were chosen to discuss these aspects with senior managers to get a broader look at the range of innovation within the school, as well as the management and leadership focus, which only those higher up in the school would be able to discuss with confidence.

The interview process will follow the questionnaires and will link in with the observations in that they will be selected from the schools chosen for the observations. 

\subsubsection{Observations}

As a teacher in a school involved in the initiative, my personal observations will be included as a data point in the research. This will be clearly identified as this is the least reliable and valuable of the methods for data collection. Observations on the implementation are useful as they provide more clarity that could be achieved from a questionnaire. My observations will be limited to the context in which I teach, the level at which my position is within the school and influenced by my own personal biases and views on innovation. 

