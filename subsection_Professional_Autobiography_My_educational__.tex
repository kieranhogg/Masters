\subsection{Professional Autobiography}
My educational background has always been heavily focused around computers and technology. My secondary education took in A-Level ICT and Computing, which naturally led me to a degree in Computer Science. During my degree I was fortunate to work for a software company for a year full time and a further year part time. This was followed by a PGCE in ICT.

Upon my qualification, English ICT qualifications were still very prescriptive and based around the teaching of Microsoft Office skills. As a result of my background, I felt naturally inclined to innovate with technology and the curriculum where it allowed. As a result, I feel fairly strongly that innovation is not the use of technology in the classroom. I believe it can be a part when used correctly, but simple use of technology is not. To take an extreme example, effective use of interactive white boards is something mentioned within this sphere, but they were developed in 1990 and by 2007, 98\% \cite{kitchen2008harnessing} of secondary schools had them in every classroom.

My natural inclination therefore is to see innovation within education as holistic rather that one particular idea. This will obviously affect my hypothesis but will have no effect on the analysis of the data relating to how other teachers' view it.