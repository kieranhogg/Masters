\subsection{The Methodology of Research and Justification}
The research is best categorised as an investigation therefore the main method of research will be a pragmatic approach. The brief handed to schools has been fairly open and the length of time schools have had to implement it is short. The expectation of large culture changes is obviously unfounded however the level to which schools have acknowledged, understood and implemented any innovation initiatives is the overall focus. 

Taking a non-pragmatic approach would lead to such a narrow set of results and conclusions that any proposals made based on them would likely not apply to the wider community. 

It is important to note of the power dynamic in this entire situation, one which is crucial to bear in mind when conducting the research, or indeed reading any results or conclusions. The innovation initiative is one that has been proposed and distributed by an external body to schools, a body which schools are otherwise held account to. Anyone participating in this research, especially anyone in a more senior leadership role, will therefore have a tendency to portray an image that perhaps be more generous that may actually be the case. To minimise this effect, the research will be conducted face-to-face where it is thought that a disclaimer or explanation would facilitate more honest dialog. Additionally, all quantative research conducted will be anonymous and where possible, independent of institution in order to gain the most accurate data feasible.





