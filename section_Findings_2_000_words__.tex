\section{Findings?}
2,000 words 

\begin{quote}
Describe and evaluate the actions you took. What helped or hindered the activities? Avoid too much narrative, but critically discuss the processes of conducting the research. Aspects such as designing a research instrument or piloting your data collection methods are appropriate to discuss referring to the research literature and how it prepared or informed you, or not, for the activity. Reflection upon your activities once they have finished will increase your learning.
\end{quote}

\begin{itemize}
\item Interviews
\item Observations
\item Questionnaires 
\end{itemize}

First in the process of data collection was the interviews with senior staff. The premise being that in theory, those positions would not only be the most informed but also have the broadest view on the topic, which might have given scope for alternative investigations. The senior staff interviews proved to be as useful as had hoped and the data collected, while qualitative, gave for some useful starting points, addressed misconceptions and laid the foundation for the rest of the research. The data from the in-person interviews was the most difficult to collate and anaylse but perhaps the richest form of data collected during the process. As expected, the challenges of finding convenient times in which to interview senior leaders proved to the greatest challenge. This was the main driving force for the use of questionnaires given their ability to case a wider net. 

An unforeseen factor in the interviews was the period of time in which the person was interviews. Early in the research process, the interviews were more structured, in a similar form to the questionnaires only with more scope for follow-up questions. As the research process developed, so did the interviews. The interviews became less structured, more exploratory and more future-looking.

Despite this, both styles were ultimately useful and the order in which they were performed proved likewise, the interview process ended up more evolutionary than structured.

Questionnaires proved to be the most useful form of data collection, just in terms of numbers collected. While impersonal and sometimes too rigid, the breadth of respondents gave insights which interviews simply could not have. There were difficulties in the preparation for the various audiences. While it was important to try to capture the same ideas and sentiments, there were inevitably multiple variations of questionnaires. The standard questionnaire was aimed at a teacher who worked inside the UAE. Very little explanation was required and the majority of the questions stood on their own. When soliciting feedback from non-UAE teachers, some of the questions had to be changed or made optional. 

As previously mentioned, the data has been supplemented with observations, these are only available at the schools personally visited. The validity of this data is good enough to be useful but not enough to be reliable as a single source. Some of the observations were retrospective, that is, they were made before this current research took place. While a useful addition, there were undoubtedly areas missed by virtue of the fact they were not being looked for. Where the observations were more recent, the data can be considered more reliable but again, it was restricted to what was easily observable, as well as the caveats mentioned in the professional biography.
