\section{Analysis}
5,000 words
\begin{quote}
Present here your results with a full discussion and interpretation. You need to analyse what your findings mean within your work context and fully interpret them, comparing them to the literature you have reviewed and other findings in your research.

Where you are presenting a lot of numerical data, it is important to include a visual representation. You can display data in, for example:
•	Tables
•	Graphs (bar, pie, scatter etc..)
•	Charts 
•	Diagrams
•	Vignettes

Immerse yourself in the data and becoming familiar with it
Code it 	identifying categories to place data into.
Themes
Patterns
Hypothesis testing
Linking- Connecting
Combine data sets: i.e. comparing observation, questioning and documentary data
Weighing: Discuss
•	Measuring/evaluating the weight of the evidence
•	Saying where there is strong and weak evidence, or none at all. 
•	Sources of bias
•	Descriptive Validity
•	Explanatory Validity
•	Reliability of the results

Link your analysis to the literature review and knowledge that has already been created. This is an important chapter in which you make sense of your results and is likely to be the longest. Ensure that you clearly cross reference to your Appendices, where appropriate, and do not just list where the evidence can be found.

Evaluate the reliability and validity of your evidence. How strong is your evidence?  How far we can rely upon your research to base our practice or are there inherent flaws? You might answer the questions: -

•	How strong is your evidence for each of your findings? (E.g. strong, some or weak evidence) 
•	How good were your research tools?
•	What are the sources of bias in your results?
•	How far is your work a valid description? (I.e. descriptive validity)
•	How secure is your analysis? 
•	If the research were done again would it produce the same results? (Reliability)
•	What difference has this research made to you as a professional and a person?
•	What difference has it made to your institution?
•	What do you intend to do next as a consequence of your research
\end{quote}

\subsection{Introduction of Innovation}
At the time of writing, less than one academic year has passed since the introduction of the new framework which highlighted innovation. Starting at the beginning, we look at the data provided from the questionnaires from teachers within the emirate. Of those who responded, 75% of replies indicated that yes, their school had made changed since the start of the academic year. 12.5% replied they had not, 12.5% were unsure. Given the varying nature of the job responsibilities for those who responded, the results are indicating that most schools have acknowledge and embraced the changes.

Of those who has replied that their school had made changes, they were invited to provide examples of the changes made. Example changes included:

\begin{itemize}
\item Subject specific innovation classes. Encouraging and demonstrating innovative teaching tools and methods
\item Innovation day (all school project), teaching with I-pad, different apps, Flipping the classroom, use of augmented reality, online assessment, etc....
\item New SEN programmes have been trialled.
\end{itemize}

One respondent replied with some examples from their inspection report which have been edited for relevancy:

\begin{itemize}
\item Students demonstrate excellent communication skills, they use these very effectively to share ideas and explain their thinking
\item Students collaborate extremely effectively
\item They can work effectively in pairs or small groups, listen attentively to each other, negotiate their responses in older year levels and make creative and thoughtful presentations
\item Children in KG are empowered in a well-structured environment to make choices and develop secure independent skills and creativity in a range of situations
\item Students in all years have acquired confident use of iPads and other digital technologies and use them in all subject areas
\item Students have a very strong work ethic when working independently and within groups, highlighted in collaborative participation in [a project], responding to STEM challenges
\item Most teachers provide opportunities for students to problem solve and respond to probing questions in planning their lessons
\item Students routinely respond confidently to ‘why’, ‘can you explain’, ‘how’ questions
\item By Years 12 and 13, students confidently ask challenging questions of each other
\item Curriculum review and modification leads to extensive opportunities for students in all years to innovate, enhance their learning and show enterprise. For example, using iPads within the continuous provision in FS and Years 1 and 2; the accelerated reader programme in primary and secondary; and Year 9 managing the complex process of publishing a high quality book of student writing
\item Senior leaders have embedded a culture of innovation into curriculum planning and development
\item Staff embrace opportunities and appreciate the support and encouragement to promote innovation within teaching and learning
\end{itemize}


While there are undoubtedly some great examples in here, it is also illuminating to note that both group work and effective questioning are highlighted as good examples of innovation. By any definition of innovation that is quite a stretch, both ideas having been cemented in good teaching practice for many years. It is an important thing to note as it does backup the theory that innovation is quite hard to define and measure, as seen by the highlighting of everyday good teaching practice as innovation by a group of experienced inspectors.

\subsection{Reaction to Introduction}

\subsection{Students}
While talking of innovation, it is quite easily to get lost in discussions of innovative teaching and innovative schools. The ultimate aim of any good school of course, is to improve the quality of education for students. It can't be said that innovation is for this end goal, some innovations are for improving the work load of teachers or the workings of schools, but a good proportion will benefit students either directly or indirectly.

At School A, where innovation weeks were used, students opinions on these weeks were solicited. The two main focii were whether students enjoyed the sessions more than usual, and whether they learnt more. The responses are shown in \cite{enjoy} and \cite{learn}

\subsection{Forms of Innovation}

\subsection{Case Studies}
During the process of interviews and questionnaires, based on the early feedback that was received, three schools were chosen to be looked at in more depth. These three schools would be categorised as above average in innovation. They are not necessarily the three most innovative schools in the area, even if that could be measured empirically, however all three have some excellent forms of innovation which made them stand out.

\subsubsection{School A}
School A is a premium\footnote{As of 2014, no later information is available for the categories https://www.adec.ac.ae/en/MediaCenter/Publications/PVT\%20Schools\%20end\%20of\%20year\%20report-\%20Irtiqaa\%20eng/HTML/files/assets/basic-html/index.html#40} adec British Curriculum School in Abu Dhabi which is relatively new school. Its inspection report was completed prior to the this academic year, resulting in no reference to innovation within it.

Whilst not directly innovation, the school focuses on hiring the best teachers available, with a rigorous hiring process which is still conducted directly by the headmaster. The impact of this on innovation is hoped that by hiring the best teachers, the level of teacher will be such that they are naturally innovative in their practice.

As with many schools, they have introduced specialised innovation weeks where students are off timetable and perform activities which are outside the usual curriculum, whether that be the activity itself, the delivery or the method of learning. The younger students participated in an innovation week which focused around Computing and Science with students taking part in more investigative approach to learning with more student-led opportunities, more group work and less structure in terms of the timetable. The older years have participated in an innovation week where mostly STEM subjects took part.

\subsubsection{School B}
School B is a premium International Baccalaureate (IB) school in Abu Dhabi. While still relatively young, it is a few years older than School B. School B has been inspected using the new inspection framework which gives greater focus on innovation. The following is the part of the reports pertaining to "Development and promotion of innovation skills":
\begin{quote}
Students demonstrate excellent communication skills. They use these very effectively to share ideas and explain their thinking. They confidently ask questions and challenge each other as, for example, when Year 8 use role play very creatively and make presentations. Students collaborate extremely effectively. They can work effectively in pairs or small groups, listen attentively to each other, negotiate their responses in older year levels and make creative and thoughtful presentations. Children in KG are empowered in a well-structured environment to make choices and develop secure independent skills and creativity in a range of situations. Students in all years have acquired confident use of iPads and other digital technologies and use them in all subject areas.

Students have a very strong work ethic when working independently and within groups. This is highlighted in collaborative participation in the secondary school Island Project, responding to STEM (science learning network) challenges, and the ‘Mantle of the Expert’ in the primary school. These develop high levels of research, problem-solving and critical thinking.

Most teachers provide opportunities for students to problem solve and respond to probing questions in planning their lessons. Students routinely respond confidently to ‘why’, ‘can you explain’, ‘how’ questions. By Years 12 and 13, students confidently ask challenging questions of each other.
Curriculum review and modification leads to extensive opportunities for students in all years to innovate, enhance their learning and show enterprise. For example, using iPads within the continuous provision in FS and Years 1 and 2; the accelerated reader programme in primary and secondary; and Year 9 managing the complex process of publishing a high quality book of student writing.

Senior leaders have embedded a culture of innovation into curriculum planning and development. Staff embrace opportunities and appreciate the support and encouragement to promote innovation within teaching and learning.
\end{quote}

In the report summary, the following was highlighted:
\begin{quote}
Students acquire key skills to innovate and be creative. They show very high understanding of environmental sustainability through the Eco club, participating in projects such as recycling, cleaning the beach or the solar-powered car challenge.

[...]

In the more effective lessons, teachers very effectively promote innovation, creativity, research and critical thinking skills. For example, use of digital technologies in all subjects and year levels to enhance research and recording. Year 9 students prepare ‘flipped’ learning videos to lead learning in art lessons; and students prepare and publish a compendium of students’ writing.

[...]

Innovative approaches to involving students in the planning, delivery and evaluation of charity events and curriculum enrichment activities develops their enterprise, innovation and collaborative skills. All students are prepared very well for their next phase
\end{quote}

As one of the better schools in the region, as well as one of the first to be evaluated using the new framework, the insight that the reports gives into how ADEC will in evaluating innovation is very useful.

One of the first things apparent in the report is what they have included under innovation. The first part relating specifically to the innovation skills contains evidence that would appear without any other context, would be difficult to be considered innovative. Collaborating, presentations and questioning are good practice and could well be part of an innovative lesson or scheme of work, but can't really be considered as evidence themselves of innovation. This over-inclusion interesting for a few reasons: firstly, this is the first glimpse of what ADEC will be looking for, which does seem to fairly broad and secondly the inspection team is not directly part of ADEC so this could perhaps explain the disconnect. They also highlighted some good use of technology which, as previously discussed, is a contentious issue when it comes to innovation, as well as a cross-curricular project which the school ran.

The cross-curricular project was rightly highlighted by both the inspection and a member of staff interviewed as a good example of innovation. Students participated in a scenario-based project over 6 weeks which encompassed Art, English, DT, Music, History, Geography, Maths, Science, ICT, Languages, Arabic and Social Studies. As well as covering a wide-range of subjects, it also gave students opportunity to work on soft skills such as communication and teamwork. 

Whilst not a pioneering idea, most Primary-trained teachers would be able to show comparative ideas in their part of the school, doing so in a Secondary school with the array of subjects involved and the time span of a half-term is definitely a good example of innovative curriculum design.

In terms of the type of innovation identified, curriculum design, teaching and learning and student innovation skills were all mentioned. Innovation within the leadership was briefly mentioned but was not quantified or explained. From interviews, it is clear that there is a clear ethos of innovation within management but perhaps an inspection does not have sufficient time to see this is sufficient depth. 

\subsubsection{School C}
School C is a premium school based on Dubai offering a mixture of British and IB curricula. In its inspection report, the following was mentioned about innovation:

\begin{quote}
The school's mission and vision aligned closely with the national innovation agenda. The strategic plan provided clear direction for further development of this highly innovative school. Innovation was an intrinsic characteristic, promoted by all stakeholders, including innovation mentors and external partners.

The building design offered diverse work areas, including open learning plazas. The learning environments included ICT that allowed students choice in both what and how they learned. Extended time blocks enabled learning that was inquiry-based, focused on thinking skills, and connected to other subjects in purposeful ways. An extensive enrichment program provided students with variety, choices and challenge.
\end{quote}

Their main focus for innovation is around learning spaces, lesson delivery and related curriculum changes. Two years ago, they introduced a large learning space which was fitted with various different arrangements of furniture 


\subsection{Opinions on Innovation}

The nature of the innovation drive, i.e. from the Government, will lead to an array of differing opinions.

\subsection{Other Themes}
* Polarising opinion on technology

\subsection{Best Practises/Case Studies}

\subsection{Future of Innovation}


What does this mean for:
\begin{itemize}
\item me
\item my school
\item other schools
\item the area/ADEC
\item worldwide
\end{itemize}

\subsection{Student Competencey Framework}
* ADEC coming in 'inspect' innovation - leads to obvious

Late in the process of this research, ADEC introduced a phased implementation of a new, but closely related, project called the Student Competency Framework (SCF). Although an ADEC project, it was authored in part by Pearson, the British publishing and education company. Similar to other competency frameworks such as RSA's, it is a set of expected skills students are meant to master. The SCF is comprised of three sets of skills, one of which being "Learning and innovation skills", defined as: "those skills which enable students to be become creative, innovative, flexible life-long learners who are able to be effective, multi-skilled productive workers". This is strong evidence that ADEC see innovation as a multi-discipline idea as they have added student innovation skills to the existing innovation areas looked at in their inspections.

While it is hard to identify specifically the origin of individual sentiments, it can be assumed that whoever wrote it, the other party would agree given the joint publication. With that said, the justification for a phased implementation is given as:

\begin{quote}
Change management in education and the implementation of innovation require careful strategic planning. Too many good initiatives have failed through limiting their roll-out strategies to a dissemination of information (Cordingley and Bell, 2007). Dissemination alone will not succeed in embedding large scale innovation in the long term. Research makes clear that a clear, structured well-integrated strategy that ensures new initiatives are embedded and then sustained by all stakeholders is required for successful delivery and implementation. Securing ‘buy-in’ is critical. This is usually achieved by ensuring that stakeholders understand and commit to the required change. This is most likely when the change is seen to benefit learners, as is the case with the SCF.
\end{quote}

This is a fairly different approach than the the one taken when adding innovation itself as an inspection criteria: the roll-out was a dissemination of information (in this case a change to the inspection framework) and there were few strategies proposed which resulted in few stakeholders understanding how to innovative. This difference will be further expanded on in a broader fashion in later chapter.

