\section{Chapter 3}
4,000 - 5,000 words
\subsection{The Methodology of Research and Justification}

\subsubsection{The purpose of the research}
The research broadly aims to investigate the surrounding areas relating to innovation within schools in the UAE. There will need to be a focus on the understanding of innovation itself, something that has come out of the literature review and preliminary research. The research will then need to look at the reaction, if any, from schools to the introduction of innovation as an inspection. Finally, some schools (if applicable) will be chosen to look at any innovative practice occurring, to attempt to produce a summary or reduction of best practice in terms of being innovative.

\subsubsection{The nature of the knowledge which is created}

\subsubsection{Types of research}

As there are multiple areas to look at within the research, it is obviously important to look at different types of research methods to ascertain the best method for data collection in each instance. In the first instance, the seminal \textit{Research Methods in Education} by \citet{Cohen2005} is the starting point to identify possible research methods that might be suitable for each task.

Throughout almost every area of the data collection, one of the key concerns was teachers, unlike many professions, do not have "free time" during the day. That is, during contact hours, a teacher could not reasonably be expected to answer a phone or reply to an email. Additionally, outside of contact hours, many teachers use that time to plan and assess student work. Teachers' time is therefore relatively costly, and asking them to take time out to provide data altruistically is a huge challenge.

The first area of research is finding out data from teachers. This encompasses looking at definitions and examples of innovation, as well as seeing if they believe their school has changed in response to the changes. As we are collecting data about the current situation, it had to be decided to choose from interviews, questionnaires, accounts, observations and personal constructs as possible methods, as taken from \citet{Cohen2005}.

The most constraining element in this area of data collection is the time (and potentially distance) needed to provide any face-to-face data collection for all but questionnaires. Needing to contact as many different people in different schools as possible meant interviews were time-prohibitive, even if they were demoted from in-person to over the phone. 


\subsubsection{The reason for your choice of research methodology}

In simple terms, the research aims to study the effect of an external influence (the innovative drive) on schools. The research methodology options are therefore already narrowed; the change has already happened and therefore does not need to be discovered or inferred. This strongly rules out the \textit{ex post facto} methodology as the cause is already known to us.

Naturalistic enquiry was included as a possible research method, with the "intention of the research is to create as vivid a reconstruction as possible of the culture or groups being studied", \cite{Cohen2005} which is not not only a possibility, but a good method of finding out the current impact of the changes. Elements of a naturalistic enquiry which seem to suggest it would be a good fit include: "researchers do not know in advance what they will see or what they will look for", "research designs emerge over time", "social research should be conducted in natual, uncontrived, real world settings with as little intrusiveness as possible by the researcher", and "data are analysed inductively, with constructs deriving from the data during the research". The idea of inductive, or iterative investigation is one that is very close the anticipated model as no-one knows what changes have happened and how people are perceiving them. 

Surveys were another possible approach as it "set[s] out to describe and to interpret \textit{what is}." \cite[p. 169]{Cohen2005} As the impact of the changes are unknown, describing the current state of those changes would be a reasonable method. 




\subsubsection{The possible measures for trusting the research (validity and reliability)}
\subsubsection{The nature and form of what may be generalised from the research.}

While the difficulty in getting accuracy in responses due the differing opinions on what innovation in education is, this will not diminish the validity of those responses, in themselves they are valuable data points. The generalisation of the research will be twofold: to what extent is innovation beneficial in schools and how best to achieve an innovative school; and reaction to a community-wide initiative and the level of success in doing so.

\subsubsection{The power relationships and involvement of stakeholders and the effects that this may have.}

It is important to note of the power dynamic in this ecosystem, one which is crucial to bear in mind when conducting the research, or indeed reading any results or conclusions. The innovation initiative is one that has been proposed and distributed by an external body to schools, a body which schools are otherwise held account to. Anyone participating in this research, especially anyone in a more senior leadership role, will therefore have a tendency to portray an image that perhaps be more generous that may actually be the case. To minimise this effect, the research will be conducted face-to-face where it is thought that a disclaimer or explanation would facilitate more honest dialogue. Additionally, all quantative research conducted will be anonymous and where possible, independent of institution in order to gain the most accurate data feasible.

\subsubsection{The ethical basis of the research.}
Ethically, the sole issue is anonymity. The relative level of success with the initiative will depends on a large number of factors, some which may not reflect positively on the staff or schools taking part. Individuals taking part have been guaranteed their anonymity as well as making the schools taking part anonymous. With a range of schools and the nature of innovations such as this, there will inevitably be some results which look less favourable for some schools. It is also important to ground this research in one very specific area which is a small part of inspection framework which means no inferences can be made on the schools in the wider context.

This section should also:
•	Identify any bias that the researcher may bring to the research and how they intend to overcome this
•	Identify any variables and how they can will be dealt with
•	Outline how your research outcomes could be generalisable
•	The Tests for Truthfulness, Reliability and Validity that may be applied
•	The Involvement and Relationships between Researcher, and Subjects

\subsection{The Methods of Data Collection and Justification for choices}
As there are multiple areas to look at within the research, it is obviously important to look at different types of research methods to ascertain the best method for data collection in each instance. In the first instance, the seminal \textit{Research Methods in Education} by \citet{Cohen2005} is the starting point to identify possible research methods that might be suitable for each task.
Throughout almost every area of the data collection, one of the key concerns was teachers, unlike many professions, do not have "free time" during the day. That is, during contact hours, a teacher could not reasonably be expected to answer a phone or reply to an email. Additionally, outside of contact hours, many teachers use that time to plan and assess student work. Teachers' time is therefore relatively costly, and asking them to take time out to provide data altruistically is a huge challenge.
The first area of research is finding out data from teachers. This encompasses looking at definitions and examples of innovation, as well as seeing if they believe their school has changed in response to the changes. As we are collecting data about the current situation, it had to be decided to choose from interviews, questionnaires, accounts, observations and personal constructs as possible methods, as taken from \citet{Cohen2005}.
The most constraining element in this area of data collection is the time (and potentially distance) needed to provide any face-to-face data collection for all but questionnaires. Needing to contact as many different people in different schools as possible meant interviews were time-prohibitive, even if they were demoted from in-person to over the phone. Accounts, observations and personal constructs, all required either a prohibitively large period of time from me, or to push the responsibility onto the teachers themselves to produce the data, both not acceptable. By reason of elimination, questionnaires were the only method where many teachers' opinions could be solicited (anticipating a level of non-completion) and the time spent both all parties would be minimised.
While the data required from teachers is mostly quantitative, the more open-ended questions should be fairly 
TODO: look at below, does it fit?

There will be three main methods of data collection for the research.

\subsubsection{Questionnaires}
The quantitative data collection of the questionnaire was an obvious choice for the main part of the research. There will be four main stakeholders participating: teachers in my school; students in my school; teachers in other Abu Dhabi schools and teachers in non-Abu Dhabi schools. 

The choice to use questionnaires in the first instance was due to the data required to be collected. The questions to how well understood and to what extent innovation had been implemented have to come from data gathered from teachers. Ideally, interviews would have been conducted in order to establish meaning, inference and to gain clarity. Due to the time constraints, questionnaires were chosen. 

The main consideration when using questionnaires is to produce a variety of open and closed questions; to avoid leading questions and to keep as many questions as possible answerable for all participants. The only caveat to this will be the questionnaire prepared for non-Abu Dhabi schools. To act as a control group, the questions must remain the same which may lead to confusion or difficultly answering depending on the situation in their school. To prevent this, an explanation for participants to simply answer to the best of their ability, or only the questions that apply to them was supplied.

\subsubsection{Interviews}

As identified by Sharma et al., a portion of their framework relates to management and leadership. Interviews were chosen to discuss these aspects with senior managers to get a broader look at the range of innovation within the school, as well as the management and leadership portion, which only those higher up in the school would be able to discuss with confidence.

\subsubsection{Observations}

As a teacher in a school involved in the initiative, my personal observations will be included as a data point in the research. This will be clearly identified as this is the least reliable and valuable of the methods for data collection. Observations on the implementation are useful as they provide more clarity that could be achieved from a questionnaire. My observations will be limited to the context in which I teach, the level at which my position is within the school and influenced by my own personal biases and views on innovation. 

