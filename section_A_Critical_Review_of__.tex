\section{A Critical Review of the Literature}
4,000 – 5,000 words

TODO: \cite{Cordingley2007}

One of the first challenges when approaching a review of the literature is to be able to define the scope of the search. In order to be able to accurately discover appropriate material, we must attempt to define innovation within the context of education. Given the simplicity of the term and the complexity of the meaning and implications, it comes as no great surprise that there is not singularly accepted definition. The best attempt to define it using the range of related definitions found during the review is as follows:

\begin{quote}
The variety of ways in which the concept of innovation has been defined by researchers reflects the nature of the discipline \cite{Gopalakrishnan_1994}, the level at which innovation is conceptualised, and whether it is being conceived as a product or process (Amabile, 1988; Kanter, 1988). According to one of the earlier definitions, innovation is an idea, practice or material artefact perceived to be new by the relevant unit of adoption \cite{Allen_1975}. Later, Anderson and King (1993) conceptualised innovation as the emergence, import or imposition of new ideas which are pursued towards implementation, through interpersonal discussion and successive remoulding of the original proposal over time. This contemporary definition not only describes the nature of innovation, but also refers to the intrinsic process of implementation. With advances in research, the concept of innovation has also been refined and a more comprehensive understanding of innovation has emerged. It may be defined as the introduction and application within a group, organisation, or wider society, of processes, products or procedures new to the relevant unit of adoption and intended to benefit the group, individual or wider society (West \& Farr, 1990) TODO: ref this.
\end{quote}  \cite{Sharma_2005}

This a more general definition than one would normally come across specifically in education, but it helps to give the full context of the derivation. The distinction of innovation being a product or a process is one that will be discussed later. \textbf{TODO: isn't discussed later}

A slightly narrower definition, although one that is probably more fitting for this context is where innovation is differentiated from enhancement by the notion that innovation is "planned deliberate change ... but it does not necessarily result in [enhancement]." \cite{hannan2002innovative}

With this innovation drive, there seems to be three main areas of primary focus: creating an innovative education system; developing innovative schools and teaching students to be innovative.

Areas that were studied during the literature review based around the following areas:
\begin{itemize}
\item Implementing innovative ideas and projects; change management
\item Innovation Skills for Students
\item Innovation Education, an Icelandic subject
\item Innovation within schools
\item Innovation within other establishments such as higher education
\item Innovation as a concept
\end{itemize}

\subsection{Innovative Education Systems}
The aim of this entire project is to enable the UAE's schools to be more innovative, to create more innovative students who will in turn become more innovative citizens in order to further the overall goal of a more innovative country, industry and economy. While the hype has died down slightly over the last few years (TODO: reference), Finland is still held up as an education system which radically transformed itself and became the go-to example for an innovative education system.

There are many studies on how and why Finland has managed to get to where it is now, one of which that identifies three key areas is \citet{Sahlberg2007}:

\begin{quote}
\textbf{Flexibility and loose standards}

Building on existing good practices and innovations in school-based curriculum development, setting of learning targets and networking through steering by information and support.


\textbf{Broad learning combined with creativity}

Teaching and learning focus on deep and broad learning giving equal value to all aspects of an individual’s growth of personality, moral, creativity, knowledge and skills.


\textbf{Intelligent accountability with trust-based professionalism}

Adoption of intelligent accountability policies and gradual building of a culture of trust within the education system that values teachers’ and headmasters’ professionalism in judging what is best for students and in reporting their learning progress.
\end{quote}

Many of the global education systems have found it difficult to emulate the Finnish model as it tends to go against most of the modern trends in education. Indeed when they list the three areas, they contrast them with the corresponding global trends:

\begin{quote}
\textbf{Standardization}

Setting clear, high and centrally prescribed performance standards for schools, teachers and students to improve the quality of outcomes.


\textbf{Focus on literacy and numeracy}

Basic knowledge and skills in reading, writing, mathematics and natural sciences as prime targets of education reform.


\textbf{Consequential accountability} 

The school performance and raising student achievement are closely tied to the processes of promotion, inspection and ultimately rewarding or punishing schools and teachers based on accountability measures, especially standardised testing as the main criteria of success.
\end{quote}

Anyone who has taught in a school in America, Europe or many other countries will recognise the latter set of policies as the norm in education systems. Is it possible the UAE could be more successful than others in its attempts to become an innovative educational system?

Some \cite{Hatherley-Greene2016} say that the UAE and the Finnish model are incompatible due to the differing social constructs, based on models from 1970s that indicate Finland to be very much more independent and egalitarian. Comparing the following, there are definitely similarities:

\begin{quote}
Emiratis tend to accept a social hierarchy and deference to figures of authority. The UAE is a collectivist society in that its members strongly associate themselves with the larger community, sometimes subsuming their personal wishes for the greater good of the group.
\end{quote} \cite{Hatherley-Greene2016}

\begin{quote}
Traditional values have endured [in Finland], these include such cultural hallmarks as a law-abiding citizenry, trust in authority, commitment to one’s social group, awareness of one’s social status and position, and a patriotic spirit.
\end{quote} \cite{Sahlberg2007}

This is echoed elsewhere: "It could be said, even at the risk of accusations of speculation, that Finnish culture still incorporates a meaningful element of the authoritarian, obedient and collectivist mentality". \cite{Simola2005}

The relatively late and rapid development of Finland, is also a process which mirrors the UAE albeit has had a slight head start:

\begin{quote}
The late process of industrialization and the simultaneous growth of the service sector brought an exceptionally rapid structural change in society. The transitions from an agricultural to an industrial society, and further to a post-industrial society, have taken place within such a short period of time that one could almost say that these societies currently co-exist in a very special way in the country.
\end{quote} \cite{Simola2005}

The countries outwardly appear to be at least superficially similar which could lead to an innovative drive in education being successful. There are of course significant differences between the two countries. Finland has previously had an extremely homogenous population, whilst migration into Finland has rapidly increased over the recent years, its rate of residents who speak a language other than Finnish is 5.4\% from a population of 5.2 million. \cite{OfficialStatisticsofFinlandOSF2011} The UAE by contrast has an estimated 1.4 million Emiratis \cite{HABBOUSH2013}from a population of 9.2 million \cite{MalitJr.2013} giving an 85\% rate of expats in the country.

\subsection{Innovation Skills for Students}
\textbf{TODO: RSA, IB}

\subsection{Innovation Studies}
Innovation Studies is an Icelandic subject which appeared in 1999, and was introduced to their national curriculum from 2007. On first reading, a subject related to Design and Technology would not appear to hold much relevance to the broader idea of innovation aside from sharing the name. Reading the emphasis behind it however, it becomes clearer as to the similarities:
\begin{quote}
[Innovation Studies] is driven by an innovation process rather than subject content and as such is cross-curricula. Students capitalize on their knowledge from all sources (Aòalnámskrá grunnskóla, 1999, 31). Ⅰn addition, innovation exercises can provide a context for researching and further understanding.
\end{quote} \cite{thorsteinsson2005innovation}

This form of innovation is therefore as a subject; a set of skills which students are allowed to develop and practice in order to become innovative students. They had also identified a subject with which to build it in to, or to develop, that being Design \& Technology.

This is an interesting facet of innovation given the other non-subject forms it are much more common, in practice and in the literature. This will be an interesting comparison in terms of how this related to teaching of student skills, whether it can be focused in a particular area or whether they need be more cross-curricular.

\subsection{Innovation Within Schools}
This is expected to be the most similar to the areas that will be focused on during the research process.

One of the more in-depth studies is a study focusing on a wide range of areas within Israeli schools. \cite{tubin2003domains} They devised a series of aspects on which they assessed each school, ultimately producing a score for the average level of innovation in each school. The schools were scored on:

\begin{itemize}
\item Physical space - use (or not) of the traditional classroom
\item Digital space - use of any technological "space" such as a VLE (Virtual Learning Environment)
\item Time - whether this was taking place within traditional lessons, virtual courses etc.
\item Student roles - developing the roles that students take within the classroom such as teaching assistants
\item Teacher roles (with students and teachers) - if teachers took on differing roles in the classroom and working with colleagues
\item Curriculum content - use of innovative techniques to enrich or change existing curricula
\item Didactic solutions - the movement away from traditional didactic approaches such as paper-based worksheets
\item Assessment methods - primarily the use of ICT to develop new or facilitate exiting assessment practices
\end{itemize}

The use of scoring to come up with an average innovation score is quite an interesting concept and the foci would seem to be a selection which most people could agree an innovative school would have some of them present. Where the study does reveal its age is the loose relationship of ICT equalling innovation; ICT is mentioned in the majority of areas looked at, with a comment on one school's innovative curriculum content stating that "the French language teacher asked her students to find their favourite French songs and food on the Internet to enrich their own Website". While this is good example of a cross-curricular lesson, this would not pass for innovation anymore.

Whilst Tubin et al. took a fairly wide look at the school environment, there are more studies which look at innovation within the management, culture and leadership of schools. One of the broader of these, but still with a focus on the process rather than the output, is \citet{Sharma_2005} which proposes a framework to maximise the impact of innovation within schools, an interesting finding being "that innovations in schools are not resource intensive". Their devised framework is:

TODO: explain these
\begin{itemize}
\item Leadership
    \begin{itemize}
    \item Support and Encouragement
    \item Networking
    \item Openness in Communication and Team Effort
    \end{itemize}
\item Review and Monitoring
\item Mobilising Community Support
\item Creating Linkages—Far and Wide
\item Teachers’ Selection, Training and Growth
\item Decentralised and Participative Management
\item Smooth Vertical and Horizontal Communication
\item Emergence of Processes and Systems Unique to the Respective Schools
\item Institutionalisation of Systems and Procedures
\end{itemize}

This is interesting to compare and contrast this framework with ADEC's own areas, with a reasonable amount of crossover:

\begin{itemize}
\item the way schools are owned, organised and managed; 
\item in curriculum design models; 
\item in teaching and learning approaches, such as the ways in which learning technologies are used; 
\item classroom design including virtual spaces; 
\item assessment; 
\item timetabling; 
\item partnerships to promote effective learning and engagement in the economy; 
\item and the ways in which teachers and leaders are recruited, trained, developed and rewarded
\end{itemize}

Sharma et al. have gone one step further than Tubin et al. by taking both a wider look at the environment that must be present for innovation to happen, as well as abstracting them into a set of principles which can be used as a framework for schools who wish to be innovative. In the absence of any guidelines set forth as to the notion of innovation, these two studies initially look like a good starting point for the investigation into the local schools and level to which they are being innovative.

\subsection{Innovation Within Other Establishments}
Innovation is an idea that has an application in almost any context. Research into innovation outside the sphere of education is certainly prevalent, and the publications discussed are those that have been identified to have some parallels or application with the target context of schools.

One of the seminal works on innovation in business is Nonaka and Takeuchi's \textit{The Knowledge-Creating Company: How Japanese Companies Create the Dynamics of Innovation} \citet{nonaka1995knowledge}. It is a wide-ranging and in-depth look at the idea that Japanese companies were (in the 1990s) seen to be extremely innovative by western observers. The breadth is far too extensive to detail and while there is some relevance to this study, the majority is understandably business-focused. While some schools are businesses too, they are unique in that their product (delivery of knowledge) is in the majority of cases, free at the point of delivery.

The core of the study is tacit and explicit knowledge; how it is used and transferred:

\begin{quote}
...[this is] a shared understanding of what the company stands for, where it is going, what kind of world it wants to live in, and, most important, how to make that world a reality. In this respect, the knowledge creating company is as much about ideals as it is about ideas. And that fact fuels innovation. The essence of innovation is to re-create the world according to a particular vision or ideal. To create new knowledge means quite literally to re-create the company and everyone in it in a nonstop process of personal and organizational self-renewal. In the knowledge-creating company, inventing new knowledge is not a specialized activity – the province of the R\&D department or marketing or strategic planning. It is a way of behaving, indeed a way of being, in which everyone is a knowledge worker – that is to say, an entrepreneur.
\end{quote}

The so-called "knowledge-creating" company is based on the "spiral of knowledge" shown in Figure \ref{label:spiral}. On reading, there is an inescapable context of product creation, invention and entrepreneurship. If that can be overlooked however, the knowledge transfer idea is a generic idea which could be applied to almost any organisation.

When thinking about schools in relation to this idea, there quickly appears the separation between teaching, teachers and students, and management, organisation and leadership. Teaching is a classic example of traditionally tacit information; teacher training definitely has explicit information such as standards and frameworks but a large part is observing, shadowing and reflecting on teaching practice. On the contrary, school management and organisation deals with explicit information, typically more than teachers have time to absorb. The process of transferring explicit information into tacit information is one which most schools might refer to as a "culture", the quicker new staff are integrated into that culture, the more effective it will be as an organisation.

\subsection{Summary}
\textbf{TODO: update in light of additions}
The main focus of investigating how schools react to being judged on being innovative is not something found in the literature. As a fairly unique situation, that was to be expected. The literature which was directly applicable was those that dealt with investigating innovative schools. Where there were gaps in the research it was that the schools were ostensibly innovative but there very little comment on how they themselves became innovative. The publication of a framework for being innovative is something which can be of use to all schools but it lacks context. It lacks the narrative of how schools got to that point, the mistakes they made, the route which they took to arrive their current position. A framework glosses over these by providing the ingredients without providing the recipe.

The literature on Innovation Studies appeared to be nothing but a similarity in phrasing on first reading. As a subject based on Design and Technology there appeared to be little in common at first. Upon further reading, it appeared to be using a fairly appropriate nomenclature as they had included quite a few innovative student skills within. The focus on prototyping and solutions to real-world problems is definitely a form of innovation. They also touched on innovation as students' skills, which was not seen a lot in other areas. Most of the school-based literature tended to focus on schools as whole; innovative teaching, including technology and innovating around other areas of school management. 



\cite{StopI4:online}