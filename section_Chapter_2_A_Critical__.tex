\section{Chapter 2 - A Critical Review of the Literature}
4,000 – 5,000 words
One of the first challenges when approaching a review of the literature is to be able to define the scope of the search. In order to be able to accurately discover appropriate material, we must attempt to define innovation within the context of education. Given the simplicity of the term and the complexity of the meaning and implications, it comes as no great surprise that there is not singularly accepted definition. The best attempt to define it using the range of related definitions found during the review is as follows:

\begin{quote}
The variety of ways in which the concept of innovation has been defined by researchers reflects the nature of the discipline \cite{Gopalakrishnan_1994}, the level at which innovation is conceptualised, and whether it is being conceived as a product or process (Amabile, 1988; Kanter, 1988). According to one of the earlier definitions, innovation is an idea, practice or material artifact perceived to be new by the relevant unit of adoption \cite{Allen_1975}. Later, Anderson and King (1993) conceptualised innovation as the emergence, import or imposition of new ideas which are pursued towards implementation, through interpersonal discussion and successive remoulding of the original proposal over time. This contemporary definition not only describes the nature of innovation, but also refers to the intrinsic process of implementation. With advances in research, the concept of innovation has also been refined and a more comprehensive understanding of innovation has emerged. It may be defined as the introduction and application within a group, organisation, or wider society, of processes, products or procedures new to the relevant unit of adoption and intended to benefit the group, individual or wider society (West & Farr, 1990).
\end{quote}  \cite{Sharma_2005}

This a more general definition than one would normally come across specifically in education, but it helps to give the full context of the derivation. The distinction of is being a product or a process \cite{} is one that will be discussed later.

A slightly more narrow definition, although one that is probably more fitting for this context is where innovation is differentiated from enhancement by the notion that innovation is "planned deliberate change ... but it does not necessarily result in [enhancement]." \cite{hannan2002innovative}

As the focus of the research is simultaneously broad (innovation is a large area of research) and narrow (specifically the implementation in the UAE) there are a wide range of topics that are generally related but very few that are close to this context. 

Areas that were studied during the literature review based around the following areas:
\begin{itemize}
\item Innovation Education - a Finnish subject
\item Innovation within the classroom
\item Innovation within the school organisation
\item Innovation within other establishments such as higher education
\ite Innovation as a concept
\end{itemize}

\subsection{Innovation Studies}
\cite{thorsteinsson2005innovation}

\cite{StopI4:online}