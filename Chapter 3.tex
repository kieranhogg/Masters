\section{Chapter 3}
4,000 - 5,000 words
\subsection{Methodology}

\subsubsection{The purpose of the research}
The aim of the research is to investigate innovation within schools in the UAE. There will be a focus on the understanding of innovation itself, something that has come out of the literature review and preliminary research. An additional areas of focus is the reaction, if any, from schools to the introduction of innovation as an inspection. Finally, some schools, if applicable, will be chosen to look at any innovative practice occurring. The aim being to produce a summary or reduction of best practice for innovation in schools.

\subsubsection{The nature of the knowledge which is created}


\subsubsection{Types of research}

As the research is multifaceted, it is obviously important to employ different types of research methods to ascertain the best method for data collection in each instance. In the first instance, the seminal \textit{Research Methods in Education} by \citet{Cohen2005} is the starting point to identify possible research methods that might be suitable for each task.

In simple terms, the research aims to study the effect of an external influence (the innovative drive) on schools. The research methodology options are therefore already narrowed; the change has already happened and therefore does not need to be discovered or inferred. This strongly rules out the \textit{ex post facto} methodology as the cause is already known to us.

Naturalistic enquiry was also included as a possible research method, as the "intention of the research is to create as vivid a reconstruction as possible of the culture or groups being studied", \cite{Cohen2005} which is not not only a possibility, but a good method of finding out the current impact of the changes. There are several elements of a naturalistic enquiry which seem to suggest it would be a good fit. Cohen states that typically, "researchers do not know in advance what they will see or what they will look for" \cite{Cohen2005} which given the area is new in UAE (and as a focus, world wide), there is no obvious starting point. Cohen follows by saying that "research designs emerge over time" and that "social research should be conducted in natual, uncontrived, real world settings with as little intrusiveness as possible by the researcher". On the subject of this method, he goes on to say: "data [is] analysed inductively, with constructs deriving from the data during the research". This idea of inductive, or iterative investigation, which constitutes adapting and evolving the research as new data emerges, is one that is very close the anticipated model as no-one knows what changes have happened and how people are perceiving them. 

The concern with naturalistic enquiry would be the lack of structure; while it seems a good fit, there is a danger that if the research designs emerging were weak, the inductive approach might break down and lead to weak research.

Surveys were another possible approach as they "set out to describe and to interpret \textit{what is}." \cite[p. 169]{Cohen2005} As the impact of the changes are unknown, describing the current state of those changes would be a reasonable method. The interpretation was a slight concern given the (anticipated) relatively small sample size and the difficulty in being in a position of strength in order to interpret or draw any strong conclusions from the data.

Action research and experiments, which in the context of education are commonly seen as classroom-based research, were thought to be too narrow and practical for this research context. An action research focus for innovation would look at an area such as the effect of innovative teaching on lessons or students. While a valuable part of the innovation framework, it would have been too narrow to focus solely on the effect in the classroom. Had the innovation focus been more developed, it might have been a useful addition, but it was thought to be too new to be able to make any useful conclusion about classroom practice.

Given the different areas of research, the decision was made to take a pragmatic approach to research which will encompass naturalistic enquiry for discovery of the influence within schools and school culture, and surveys to find out the measurable data within the changes.  A pragmatic approach fits well with this attempt to find the holistic effect of innovation within schools. According to Johnson, pragmatic research "[v]iews current truth, meaning, and knowledge as tentative and as changing over time" \cite[p. 18]{Johnson_2004}. Nowhere more is this important than in the early stages of this innovation drive. As schools digest and develop the changes and inspectors make their first inspections, the understanding of innovation in schools may well change.

Johnson goes on to state that this research method "[e]ndorses eclecticism and pluralism (e.g., different, even conflicting, theories and perspectives can be useful; observation, experience, and experiments are all useful ways to gain an understanding of people and the world)" \cite[p. 18]{Johnson_2004}. It was identified early on that defining innovation and what may constitute it in schools would be an area of focus. Researching the multiplicity of innovation is more useful in this context than discarding data because not everyone may agree on what it is. The conflicting data is in itself a useful data point.


\subsubsection{The reason for your choice of research methodology}



\subsubsection{The possible measures for trusting the research (validity and reliability)}
While the difficulty in obtaining accuracy in responses due the differing opinions on what innovation in education is, this will not diminish the validity of those responses, in themselves they are valuable data points. 

\subsubsection{The nature and form of what may be generalised from the research.}
Making innovation a deliberate focus within an education system is something that has never been consciously attempted before. By researching this area, the results will usable to decide to what extent innovation is beneficial in schools and how best to achieve an innovative school. The research will be generalisable on a small school in the context of using the data from any schools who have been found to be excelling in providing an innovative education. It will also be generalisable in the wider sense of trying to improve a nation's education system through a focus on innovation. 


\subsubsection{The power relationships and involvement of stakeholders and the effects that this may have.}

It is important to note of the power dynamic in this ecosystem, one which is crucial to bear in mind when conducting the research, or indeed reading any results or conclusions. The innovation initiative is one that has been proposed and distributed by an external body to schools, a body which schools are otherwise held account to. Anyone participating in this research, especially anyone in a more senior leadership role, will therefore have a tendency to portray an image that may perhaps be more generous that may actually be the case. The responsibility of implementing these changes will fall to senior leaders, so any self-provided evidence of progress or success will need to be validated by multiple parties.

To minimise the effect of exaggeration on successes, the research will be conducted face-to-face where it is thought that a disclaimer or explanation would facilitate more honest dialogue. Additionally, all quantitative research conducted will be confidential, institutions will be anonymised  and independent of institution in order to gain the most accurate data feasible. 

\subsubsection{The ethical basis of the research.}
Ethically, the main concern is confidentiality and anonymity. The relative level of success with the initiative will depend on a large number of factors, some which may not reflect positively on the staff or schools taking part. For the questionnaire to be most useful, their schools will have to be identifiable to the researcher to enable correlation. For this reason, it is not possible for the participants to be completely anonymous, they will however be completely confidential. Any identifying features from individual results will be removed and schools will be anonymised. 

With a range of schools and the nature of innovations such as this, there will inevitably be some results which look less favourable for some schools. It is also important to ground this research in one very specific area, innovation, which means no inferences can be made on the schools in the wider context. If a school is found not to have reacted to the changes, there is no reflection on the quality of the school overall.

\subsubsection{Identify any bias that the researcher may bring to the research and how they intend to overcome this}

The author has made his own biases known in a separate section and discussed the ways to mitigate those.

\subsubsection{Identify any variables and how they can will be dealt with}

\subsection{Methods of Data Collection}
Having selected a pragmatic or mixed methods research methodology, the data collection methods chosen are questionnaires, interviews and observations.

Throughout almost every area of the data collection, one of the key concerns was that teachers, unlike many professions, do not have "free time" during the day. That is, during contact hours, a teacher could not reasonably be expected to answer a phone or reply to an email. Additionally, outside of contact hours, many teachers use that time to plan and assess student work. Teachers' time is therefore relatively costly, and asking them to take time out to provide data altruistically is a huge challenge.

The first area of research is finding out data from teachers. This encompasses looking at definitions and examples of innovation, as well as seeing if they believe their school has changed in response to the innovation drive. As we are collecting data about the current situation, it had to be decided to choose from interviews, questionnaires, accounts, observations and personal constructs as possible methods, as taken from \citet{Cohen2005}.

The most constraining element in this area of data collection is the time (and potentially distance) needed to provide any face-to-face data collection for all but questionnaires. Needing to contact as many different people in different schools as possible meant interviews were time-prohibitive, even if they were conducted over the phone. Accounts, observations and personal constructs, all required either a prohibitively large period of time from the researcher, or to push the responsibility onto the teachers themselves to produce the data, both choices were not acceptable. By reason of elimination, questionnaires were the only method where many teachers' opinions could be solicited (anticipating a level of non-completion) and the time spent both all parties would be minimised. 

While the data required from teachers is mostly quantitative, the more open-ended questions allowed for a degree of freedom in expressing their thoughts both on innovation, and to their school's 

There will be three main methods of data collection for the research, a mix of qualitative and quantitative data collection which comprise the chosen pragmatic approach.

\subsubsection{Questionnaires}

The focus for the questionnaire was to find out how respondents defined innovation and what, if any, innovation was taking place in their school. Teachers both in and out of the UAE were chosen to send the questionnaires to. While innovation is obviously a focus in UAE schools, it was thought useful to be able to compare other schools' level of innovation. The same questionnaire was used for both-UAE and non-UAE teachers. 

The choice to use questionnaires in the first instance was due to the nature of the data required to be collected. The questions to how well understood and to what extent innovation had been implemented have to come from data gathered from teachers. Ideally, interviews would have been conducted in order to establish meaning, inference and to gain clarity. Due to the time constraints, questionnaires were chosen. 

The main consideration when using questionnaires is to produce a variety of open and closed questions; to avoid leading questions and to keep as many questions as possible answerable for all participants. For the results to be comparable, the same questionnaire was used for both-UAE and non-UAE teachers.  As this may have lead to confusion or difficultly answering depending on the situation in their school, an explanation for participants to simply answer to the best of their ability, or only the questions that apply to them was supplied.

In planning the questionnaire, the first step was to produce a list of questions it needed to answer:

\begin{enumerate}
\item What is innovation (in the respondent's opinion)
\item What, if any, innovation is taking place in their school?
\item Has there been any change since the start of the academic year in relation to innovation?
\item What impact does innovation have on teachers and students?
\end{enumerate}

These questions were then expanded on to provide some more quantitative data such as which areas they thought their school was innovative in and their role in the school. Qualitative data collection opportunities were included with questions allowing the respondent to expand on good and bad examples or innovation, both in and out of their school.

The final questionnaire used for teachers is shown in appendix \ref{appendix:teachersquestions}.

The questionnaire was sent to a wide range of schools electronically, as an online questionnaire. They were distributed in two ways: the first set were emailed to as many schools in Abu Dhabi as I could find an email address for. They were addressed to someone in charge of teaching and learning with an explanation and request to be distributed within their school. The second method was to send an email to every incoming member of staff for the forthcoming academic year at my school. As an international school, the incoming cohort was from a wide range of backgrounds, locations and positions.


\subsubsection{Interviews}

As identified by Sharma et al. \citep{Sharma2005}, a portion of their framework relates to management and leadership. ADEC also included leadership and management as an area of innovation. Interviews were chosen to discuss these aspects with senior managers to get a broader look at the range of innovation within the school, as well as the management and leadership portion, which only those higher up in the school would be able to discuss with confidence.

One of areas with the potential to undermine the reliability and validity of data gathered during interviews is the phrasing of questions, specifically the use of leading questions. To this end, the interviews consisted of an introduction about the context of the research, followed by few, but open, general questions. Interviews were designed to be investigative rather then structured, so a complete list of questions was not suitable, but every interview revolved around questions such as: "what is innovation?", "what are your thoughts on innovation as an focus?", "is the school doing anything differently this year (in terms of innovation)?" and "what do you see in the future regarding the innovation focus?". The aim was to discover the thoughts and opinions of the leaders interviewed rather than to produce quantitative data, which is why open and general questions were used.

\subsubsection{Observations and Case Studies}

A 'thick description' "refers to the researcher’s task of both describing and interpreting observed social action (or behavior) within its particular context" \cite[p. 543]{thickdescription}. An important part of a pragmatic methodology is discovering the context in which research is taking place. In order to discover the 'thick description' within schools, observational case studies were chosen to provide this context.

The case study schools were not due to be pre-selected, but were planned to be identified based on the responses to the questionnaires. Whether or not schools were adapting the changes was an important data point of the research, but finding good examples based on the questionnaires would be useful to schools in the area. Schools were to be identified via the questionnaire responses as innovative, with a broad range of innovative practice sought to study in further detail.

This extra layer of research on top of the questionnaires is yet again due to seeking the maximum context for the research. While a small amount of context can be given in questionnaires, there would not be enough for a 'thick description'. Case studies "can establish cause and effect, indeed one of their strengths is that they observe effects in real contexts, recognizing that context is a powerful determinant of both causes and effects" \cite[p. 181]{Cohen2005}.

The drawbacks of the case study is that it can be too localised or contextualised to be useful to be generalisable. The impact of this is planned to be mitigated by the fact it the good practice is aiming to be itself abstracted away to provide the most useful generalisation, as well as including this as a part of the greater mixed methodology.
