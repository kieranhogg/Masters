\subsection{The Methods of Data Collection and Justification for choices}
There will be three main methods of data collection for the research.

\subsubsection{Questionnaires}
The quantitative data collection of the questionnaire was an obvious choice for the main part of the research. There will be four main stakeholders participating: teachers in my school; students in my school; teachers in other Abu Dhabi schools and teachers in non-Abu Dhabi schools. 

The choice to use questionnaires in the first instance was down the data required to be collected. The questions to how well understood and to what extent innovation had been implemented have to come from data gathered from teachers. Ideally, interviews would have been conducted in order to establish meaning, inference and to gain clarity. Due to the time constraints, questionnaires were chosen. 

The main consideration when using questionnaires is to produce a variety of open and closed questions; to avoid leading questions and to keep as many questions as possible answerable for all participants. The only caveat to this will be the questionnaire prepared for non-Abu Dhabi schools. To act as a control group, the questions must remain the same which may lead to confusion or difficultly answering depending on the situation in their school. To prevent this, an explanation for participants to simply answer to the best of their ability, or only the questions that apply to them was supplied.

\subsubsection{Interviews}

As identified by Sharma et al., a portion of their framework relates to management and leadership. Interview were chosen to discuss 

\subsubsection{Observations}

As a teacher in a school involved in the initiative, my personal observations will be included as a data point in the research. This will be clearly identified as this is the least reliable and valuable of the methods for data collection. Observations on the implementation are useful as they provide more clarity that could be achieved from a questionnaire. My observations will be limited to the context in which I teach, the level at which my position is within the school and influenced by my own personal biases and views on innovation. 