\section{Methodology}
4,000 - 5,000 words
\begin{quote}
This section should contain your plan for the collection of data which should be systematic and realistic, taking into account any variables or problems you may come across. It should also include how you are going to triangulate your findings.

Discuss the tools you will use to collect your data covering the following:

Collecting data using observation
Will observation be a strategy you will choose to use as part of your data collection process?
If so:
1.	Will you be the best person to undertake the observation or is it that you want to be observed?  If the latter, who could you ask to be the observer?
2.	Who will you choose to observe?
3.	How will you select people to be observed?
4.	What will you be observing?  Try to be specific.
5.	Do you need the co-operation of other people in order to observe?  For example, another class teacher.
6.	When and where will the observations take place?
7.	How long will you observe for?
8.	How will you record your observations?
9.	Who has access to your observations?
10.	Have you had to produce your own tool or are you using a published one?
11.	Have you had a trial run and tinkered with the schedule

Collecting data using interviews
1.	How many interviews will you decide to conduct?
2.	Who will you want to interview and why?
3.	How will you decide whom to interview?
4.	What kind of interview will you conduct - e.g. a formal interview with set questions, a more informal interview with some questions but space for the interviewee to give you a more personal reflection or a completely unstructured interview around a given topic
5.	How will you record your interviews?
6.	How will you need to transcribe the interviews?
7.	How you will interpret your interviews
	
Collecting data using questionnaires
1.	What questions do you need answering for your research?
2.	What types of questionnaire design are there?
3.	Why have you selected the method you have chosen?
4.	How do you know that you are asking the questions in a way that will elicit the types of response needed?  For example, are you going to ask people to tick boxes, give written responses, answer on a scale, etc.?
5.	Who will you ask to complete your questionnaires?
6.	Are respondents going to be identifiable or will the returns be anonymous?
7.	How will you collate the responses?
8.	How many questionnaires will you need to give out and receive back to make a viable sample?
9.	How will you select the sample.


Artefacts etc..
Outline any artefacts that will help you tell the story of your focus and how you will analyse and use them. State why you have selected them. These may include: -
	Learning materials
	Photographs
	Schemes of work
	School policies
	Children's work (written or otherwise)
	Minutes of meetings
	Letters from parents
	Ofsted reports
	School review reports
	Outcomes from performance management processes
	School Plan
\end{quote}

\subsection{The Methodology of Research and Justification}
The overall purpose of the research is to the find the extent, if any, the focus on innovation in Abu Dhabi schools has changed schools. If schools have changed, what has changed and has there been any observable impact?

As previously discussed, the idea of innovation as a stand-alone concept, and more specifically, when placed in an educational context is not one that has an agreed definition. This leads to the obvious problem of asking people to discuss an area they themselves may not be completely clear on the definition. Rather than circumvent this issue by providing a context in which the research is conducted into, it has itself been included.

One of the areas for research is how well the idea has itself been understood, and to what level of uniformity has been achieved when developed innovation within their own schools. This could lead to more variance in the data received which make an assumption that the subject knows what innovation is, but the trade-off is made to get a clearer overall picture.

\subsubsection{The purpose of the research}
The research is best categorised as an investigation therefore the main method of research will be a pragmatic approach. The brief handed to schools has been fairly open and the length of time schools have had to implement it is short. The expectation of large culture changes is obviously unfounded however the level to which schools have acknowledged, understood and implemented any innovation initiatives is the overall focus. 

Taking a non-pragmatic approach would lead to such a narrow set of results and conclusions that any proposals made based on them would likely not apply to the wider community. 

It is important to note of the power dynamic in this ecosystem, one which is crucial to bear in mind when conducting the research, or indeed reading any results or conclusions. The innovation initiative is one that has been proposed and distributed by an external body to schools, a body which schools are otherwise held account to. Anyone participating in this research, especially anyone in a more senior leadership role, will therefore have a tendency to portray an image that perhaps be more generous that may actually be the case. To minimise this effect, the research will be conducted face-to-face where it is thought that a disclaimer or explanation would facilitate more honest dialog. Additionally, all quantative research conducted will be anonymous and where possible, independent of institution in order to gain the most accurate data feasible.

\subsubsection{The nature of the knowledge which is created}
\subsubsection{Types of research}
\subsubsection{The reason for your choice of research methodology}
\subsubsection{The possible measures for trusting the research (validity and reliability)}
\subsubsection{The nature and form of what may be generalised from the research.}

While the difficulty in getting accuracy in responses due the differing opinions on what innovation in education is, this will not diminish the validity of those responses, in themselves they are valuable data points. The generalisation of the research will be twofold: to what extent is innovation beneficial in schools and how best to achieve an innovative school; and reaction to a community-wide initiative and the level of success in doing so.

\subsubsection{The power relationships and involvement of stakeholders and the effects that this may have.}
\subsubsection{The ethical basis of the research.}
Ethically, the sole issue is anonymity. The relative level of success with the initiative will depends on a large number of factors, some which may not reflect positively on the staff or schools taking part. Individuals taking part have been guaranteed their anonymity as well as anonymising the schools taking part. With a range of schools and the nature of innovations such as this, there will inevitably be some results which look less favourable for some schools. It is also important to ground this research in one very specific area which is a small part of inspection framework which means no inferences can be made on the schools in the wider context.

This section should also:
•	Identify any bias that the researcher may bring to the research and how they intend to overcome this
•	Identify any variables and how they can will be dealt with
•	Outline how your research outcomes could be generalisable
•	The Tests for Truthfulness, Reliability and Validity that may be applied
•	The Involvement and Relationships between Researcher, and Subjects



