\section{Chapter 6}
\subsection{Conclusions and recommendations}
Abu Dhabi and the UAE have set out to drive forward its education system by way of making it more innovative. The focus on innovation within UAE’s society was included as an important facet of their UAE 2023 Vision document. TODO: check this. In order for the UAE to produce more innovation citizens, the obvious knock-on effect was to include this as a component of the education system. It was included as a relatively large component of the UAE unified inspection framework for the academic year 2015-2016.

In the first instance, there was an initial surprise to the addition from schools, a reaction which was discovered through observations and interviews. While educational themes and foci come, and frequently go, with time in most countries, the introduction of something as abstract as innovation, as well and the prominence given to it was one that was new to most.

For many schools, a rear-guard reaction to this was for many schools to plan innovation days or weeks early in the 2015-2016 academic year. Some proactively, but many in response to an announced Emirate-wide innovation week 22nd-28th November. 

The innovation weeks highlighted two problems that would become evident at other times: it was announced with short notice and as a result, it conflicted with many schools’ academic plans, including internal and external exams.
Early on in the process, it highlighted that innovation takes adequate planning, as well as the fact it can often conflict with daily school life if arranged as an off-timetable day or week.

Observations and interviews reflecting on this time showed a great deal of uncertainty, especially those schools who were due to be inspected that academic year. Defining innovation and identifying suitable examples was the first challenge within schools. The innovation week, while short notice, did provide an indication that showcasing innovative teaching and learning opportunities would be the focus of ADEC and its inspectors.

During the research, a range of teachers were interviewed and a few schools were chosen to look into more closely in the form of a case study. This was combined with personal observation of how the impact was in a school.

After a few terms of having innovation in the framework, teachers could provide reasonably clear definitions of what innovation is. The definitions fell into two general categories, those that discussed innovation with respect to teaching and learning, and those that were more general, definition innovation more as a whole-school process.

This still indicates that while innovation is understood as a principle, there are multiple strands of innovation that while fall under the umbrella definition, which look very different in practice. Those definitions referring to whole-school innovation did also not come from just management as might have been expected.

It is not a surprise that the strands of innovation in education have become conflated, innovation is a process which can be applied to almost every area as identified in the opening of this document. Indeed, ADEC themselves provided eight areas of possible innovation, many of those containing multiple smaller areas.

When looking at case studies of schools which were above average in terms of innovation, it became clear that the more innovative schools are tackling many of the strands and that innovation is a school focus which is started with the leadership team and disseminated effectively throughout the school. There were many methods of doing this: giving staff time or money to focus on discovery of innovative ideas or pilot schemes; encouragement or allowances to facilitate greater collaboration with an aim at innovative learning events and a focus on innovation within teaching and learning.

It is very difficult, if not impossible, to become an innovative school, or to produce innovative schools without a focus and change in priorities at, or near, the top of a school leadership structure. As we have already discussed, innovation at a micro level can often lead to failure. If there is not a culture of trust wherein teachers are not worried about failing when innovating, it cannot succeed at the macro, whole-school (or wider) level.

There was also an element of experimentation and action research seen, which makes sense if you treat innovation in a systematic way by trying and evaluation multiple ideas and projects to rapidly iterate to find the better ideas. This is a promising if we are to take the Finnish model as one that could be replicated; their intensive five-year teaching qualification is heavily focused on producing teachers for whom educational theory and research is an utmost priority. 

Outside of the schools themselves, it was clear from the literature, that the innovation focus is here to stay with it being part of the national agenda. This brought up a few opportunities as well as a few challenges.

As a relatively young country and along with the governance structure and emphasis on education, the foundation of becoming an innovative education on a global level is there. Parallels with the Finnish education system were drawn, especially in terms of social and industrial development. One major difference, and one highlighted in the development of the Finnish model was the lack of cultural diversity within the country, albeit it had increased rapidly during this time. The cultural diversity in the UAE is rightfully seen as a cultural and social success, it does however lead to additional challenges in gaining any consistency withing teaching standards; the teaching body consists of teachers from many different countries, with their inherent differences in qualification standards that brings.

This is highlighted by one of the outcomes from the case studies: innovation works best when built on a foundation of solid teaching and learning. For all the strides the UAE has made in recent years, its education is still ranked as 46th out of 65 based on the 2012 PISA TODO:REF rankings. They are however, aimed for top 20 by 2021. TODO: does this read as advocacy?

One of the problems identified in UAE public schools (TODO: WHERE?) is the use of rote teaching practices and teaching form textbooks. This is backed up in an interview by the OECD who explains that the UAE is “way below” expectations, mainly because of the lack of creative thinking from students, which the PISA ostensibly tests.

TODO: ref UAE’s PISA scores “way below” expectations says OECD head.
An innovation drive is therefore a good solution to close this gap, but without a solid teaching and learning platform, innovation can fall short and could even exacerbate the problem for some schools.

There also is the problem of not having central control over school curricula. One of the components identified in the success of the Finnish improvement was a central and innovative curriculum that all schools taught from (TODO, check and ref). One of the ostensible advantages of schools in Abu Dhabi is the ability to choose from multiple curricula, British, American, French, Japanese, Indian and local curricula comprising the majority of schools. This enables families in a transient country to choose a school curriculum that is either familiar to them, or one that will provide continuity if they regularly move or if students are aiming for university. The advantage can also prove a disadvantage when trying to enforce consistency and a common direction.

Innovative teaching is often seen as the antithesis to rote-style learning but many of the curriculums are still reliant on it, primarily due to the prominence, amount or form of formal assessments. This reflects the greater educational conundrum of trying to teach a wide and engaging curriculum while being assessed as a school on their national examination results. Quite how much pressure and expectation is placed on international schools compared to their local counterparts is outside the scope of this research but could play a factor in how well the innovation drive performs.

As project in itself, aiming to create an innovative education system, is innovative. The introduction of the SCF late in the academic year provided an interesting addendum to the entire process. Firstly, because it goes hand-in-hand with the existing innovation drive and secondly, because the SCF document provides a blueprint for change management of innovative projects, a blueprint which could be equally applied to itself, school projects and the innovation drive itself.

It is hard to convincingly argue that the innovation inclusion in framework is the same size as the SCF project, and therefore would benefit from being treated in the same manner. However, once we compare the issues raised by teachers and schools during the first year with the research in the SCF document the justification becomes a bit stronger.

Using the innovation implementation (TODO: REF), it indicates that there was a lack of clear vision for the innovation drive. Without confirmation, it can only be speculated if the SCF is a sister project  of the innovative drive, something which falls underneath it as a strand, or a replacement.

The inclusion of innovation within the unified framework is a solid idea, with it seemingly addressing a real need in helping the UAE achieve a better education and industry. It has been shown that a clearer vision would have helped prevent the initial confusion, specifically some context or global comparisons would have helped schools, the SCF project has done that in a completely thorough manner.

The UAE’s development of education in the previous decade has been rapid and mostly successful, although signs of cohesion are emerging, there is still a way to go to achieve any standard of consistency. The diversity of educational establishments has been both an advantage to its large expatriate community for choice, as well as potential challenge to the government to ensure foci such as this one provide the greatest impact.

The relative disparity in the standard of teaching from the worst to the best schools is still large. It was indicated that innovation would have the most impact on the better schools, which could lead to a negative opinion of innovation in other schools, wherein it could become a stick rather than a carrot. (TODO too colloquial?)

The UAE has all the components it needs to create a world-class education system in the same way Dubai has become the global city it has. The national agenda of innovation is a clear path to address its current problems with its nationally-underperforming education system. While it is still early, it is already showing some signs of promise in schools. 

Ultimately it will be up to schools themselves whether innovation can become permeated within the fabric of their institutions as a driving force for cutting edge education, or whether it will be something to wheel out for the inspectors only to put it back in the cupboard to gather dust until the next inspection.

GENERAL TODO
\begin{itemize}
\item Teacher qualification – degree, vs Finland, qualifications
\item Other global initiatives, e.g. UK focus on X
\item Abu Dhabi vs UAE throughout
\item Impact of Teaching inspections on innovation
\item Wording of “Innovation Drive”
\item Cranleigh community
\item Add page references
\item Non-UAE responses
\item check tenses

\end{itemize}

\subsection{Appendices}

