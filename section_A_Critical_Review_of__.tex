\section{A Critical Review of the Literature}
4,000 – 5,000 words
One of the first challenges when approaching a review of the literature is to be able to define the scope of the search. In order to be able to accurately discover appropriate material, we must attempt to define innovation within the context of education. Given the simplicity of the term and the complexity of the meaning and implications, it comes as no great surprise that there is not singularly accepted definition. The best attempt to define it using the range of related definitions found during the review is as follows:

\begin{quote}
The variety of ways in which the concept of innovation has been defined by researchers reflects the nature of the discipline \cite{Gopalakrishnan_1994}, the level at which innovation is conceptualised, and whether it is being conceived as a product or process (Amabile, 1988; Kanter, 1988). According to one of the earlier definitions, innovation is an idea, practice or material artefact perceived to be new by the relevant unit of adoption \cite{Allen_1975}. Later, Anderson and King (1993) conceptualised innovation as the emergence, import or imposition of new ideas which are pursued towards implementation, through interpersonal discussion and successive remoulding of the original proposal over time. This contemporary definition not only describes the nature of innovation, but also refers to the intrinsic process of implementation. With advances in research, the concept of innovation has also been refined and a more comprehensive understanding of innovation has emerged. It may be defined as the introduction and application within a group, organisation, or wider society, of processes, products or procedures new to the relevant unit of adoption and intended to benefit the group, individual or wider society (West & Farr, 1990).
\end{quote}  \cite{Sharma_2005}

This a more general definition than one would normally come across specifically in education, but it helps to give the full context of the derivation. The distinction of is being a product or a process \cite{} is one that will be discussed later.

A slightly more narrow definition, although one that is probably more fitting for this context is where innovation is differentiated from enhancement by the notion that innovation is "planned deliberate change ... but it does not necessarily result in [enhancement]." \cite{hannan2002innovative}

As the focus of the research is simultaneously broad (innovation is a large area of research) and narrow (specifically the implementation in the UAE) there are a wide range of topics that are generally related but very few that are close to this context. 

Areas that were studied during the literature review based around the following areas:
\begin{itemize}
\item Innovation Skills for Students
\item Innovation Education - a Icelandic subject
\item Innovation within schools
\item Innovation within other establishments such as higher education
\ite Innovation as a concept
\end{itemize}

\subsection{Innovation Skills for Students}
TODO: RSA, IB

\subsection{Innovation Studies}
Innovation Studies is an Icelandic subject which appeared in 1999, and was introduced to their national curriculum from 2007. On first reading, a subject related to Design & Technology would not appear to hold much relevance to the broader idea of innovation aside from sharing the name. Reading the emphasis behind it however, it becomes clearer as to the similarities:
\begin{quote}
[Innovation Studies] is driven by an innovation process rather than subject content and as such is cross-curricula. Students capitalize on their
knowledge from all sources (Aòalnámskrá grunnskóla, 1999, 31). Ⅰn addition innovation exercises can provide a context for researching and further understanding.
\end{quote} \cite{thorsteinsson2005innovation}

This form of innovation is therefore as a subject; a set of skills which students are allowed to develop and practice in order to become innovative students. They had also identified a subject with which to build it in to, or to develop, that being Design & Technology.

This is an interesting facet of innovation given the other non-subject forms it are much more common, in practice and in the literature. I think this will be an interesting 

\subsection{Innovation Within Schools}
This is expected to the be the most similar to the areas that will be focused on during the research process.

One of the more in-depth studies is a study focusing on a wide range of areas within Israeli schools. \cite{tubin2003domains} They devised a series of aspects on which they assessed each school, ultimately producing a score for the average level of innovation in each school. The schools were scored on:

\begin{itemize}
\item Physical space - use (or not) of the traditional classroom
\item Digital space - use of any technological "space" such as a VLE (Virtual Learning Environment)
\item Time - whether this was taking place within traditional lessons, virtual courses etc.
\item Student roles - developing the roles that students take within the classroom such as teaching assistants
\item Teacher roles (with students and teachers) - if teachers took on differing roles in the classroom and working with colleagues
\item Curriculum content - use of innovative techniques to enrich or change existing curricula
\item Didactic solutions - the movement away from traditional didactic approaches such as paper-based worksheets
\item Assessment methods - primarily the use of ICT to develop new or facilitate exiting assessment practices
\end{itemize}

The use of scoring to come up with a average innovation score is quite an interesting concept and the foci would seem to be a selection which most people could agree an innovative school would have some of them present. Where the study does reveal its age is the loose relationship of ICT equaling innovation. ICT is mentioned in the majority of areas looked at, with a comment on one school's innovative curriculum content stating that "the French language teacher asked her students to find their favorite French songs and food on the Internet to enrich their own Website". While a good example of a cross-curricular lesson, that would not pass for innovation anymore.

Whilst Tubin et al. took a fairly wide look at the school environment, there are more studies which look at innovation within the management, culture and leadership of schools. One of the broader of these, but still with a focus on the process rather than the output is \cite{Sharma_2005} which proposes a framework to maximise the impact of innovation within schools, an interesting finding being "that innovations in schools are not resource intensive". Their devised framework is:

TODO: explain these
\begin{itemize}
\item Leadership
    \begin{itemize}
    \item Support and Encouragement
    \item Networking
    \item Openness in Communication and Team Effort
    \end{itemize}
\item Review and Monitoring
\item Mobilising Community Support
\item Creating Linkages—Far and Wide
\item Teachers’ Selection, Training and Growth
\item Decentralised and Participative Management
\item Smooth Vertical and Horizontal Communication
\item Emergence of Processes and Systems Unique to the Respective Schools
\item Institutionalisation of Systems and Procedures
\end{itemize}

This is interesting to compare and contrast this framework with ADEC's own:

\begin{itemize}
\item the way schools are owned, organised and managed; 
\item in curriculum design models; 
\item in teaching and learning approaches, such as the ways in which learning technologies are used; 
\item classroom design including virtual spaces; 
\item assessment; 
\item timetabling; 
\item partnerships to promote effective learning and engagement in the economy; 
\item and the ways in which teachers and leaders are recruited, trained, developed and rewarded
\end{itemize}

TODO: format this
\begin{center}
    \begin{tabular}{ | l | p{5cm} |}
    Tubin et al.  & ADEC \\ \hline
    Leadership & the way schools are owned, organised and managed;  \\ \hline
    Review and Monitoring  &  \\ \hline
    Mobilising Community Support &  partnerships to promote effective learning and engagement in the economy; \\ \hline
    Creating Linkages—Far and Wide &  partnerships to promote effective learning and engagement in the economy; \\ \hline
    Teachers’ Selection, Training and Growth & the ways in which teachers and leaders are recruited, trained, developed and rewarded \\ \hline
    Decentralised and Participative Management &  \\ \hline
    Smooth Vertical and Horizontal Communication &  \\ \hline
    Institutionalisation of Systems and Procedures &  \\ \hline
     & curriculum design models \\ \hline
     & in teaching and learning approaches, such as the ways in which learning technologies are used \\ \hline
     & classroom design including virtual spaces \\ \hline
     & assessment \\ \hline
     & timetabling \\ \hline
    \end{tabular} 
\end{center}


Sharma et al. have gone one step further than Tubin et al. by taking both a wider look at the environment that must be present for innovation to happen, as well as abstracting them into a set of principles which can be used as a framework for schools who wish to be innovative. In the absence of any guidelines set forth as to the notion of innovation, these two studies initially looks like a good starting point for the investigation into the local schools and level to which they are being innovative.

\subsection{Innovation Within Other Establishments}
Innovation, as previously discussed, is an idea that has application in almost any context. Research into innovation outside the sphere of education is certainly prevalent, and the publications discussed are those that have been identified to have some parallels or application with the target context of schools.

One of the seminal works on innovation in business is Nonaka and Takeuchi's \textit{The Knowledge-Creating Company: How Japanese Companies Create the Dynamics of Innovation} \cite{nonaka1995knowledge}. It is a wide-ranging and in-depth look at the idea that Japanese companies were (in the 1990s) seen to be extremely innovative by western observers. The breadth is far too extensive too detail and while there is some relevance to this study, the majority is understandably business-focused. While schools are businesses too, they are fairly unique in that their product (delivery of knowledge) is in the majority of cases, free at the point of delivery.

The core of the study is tacit and explicit knowledge; how it is used and transfered:

\begin{quote}
...[this is] a shared understanding of what the company stands for, where it is going, what kind of world it wants to live in, and, most important, how to make that world a reality. In this respect, the knowledge creating company is as much about ideals as it is about ideas. And that fact fuels innovation. The essence of innovation is to re-create the world according to a particular vision or ideal. To create new knowledge means quite literally to re-create the company and everyone in it in a nonstop process of personal and organizational self-renewal. In the knowledge-creating company, inventing new knowledge is not a specialized activity – the province of the R&D department or marketing or strategic planning. It is a way of behaving, indeed a way of being, in which everyone is a knowledge worker – that is to say, an entrepreneur.
\end{quote}

The so-called "knowledge-creating" company is based on the "spiral of knowledge". \ref{Spiral of Knowledge} On reading, there is an inescapable context of product creation, invention and entrepensurship. If that can be overlooked however, the knowledge transfer idea is a generic idea which could be applied to almost any organisation.

When thinking about schools in relation to this idea, their quickly appears the separation between teaching, teachers and students, and management, organisation and leadership. Teaching is a classic example of traditionally tacit information; teacher training definitely has explicit information such as standards and frameworks but a large part is observing, shadowing and reflecting on teaching practise. On the contrary, school management and organisation deals with explicit information, typically more than teachers have time to absorb. The process of transferring explicit information into tacit information is one which most schools might refer to as a "culture", the quicker new staff are integrated into that culture, the more effective it will be as an organisation.


\subsection{Summary}
The main focus of investigating how schools react to being judged on being innovative is not something found in the literature. As a fairly unique situation, that was to be expected. The literature which was directly applicable was those that dealt with investigating innovative schools. Where it lacked was that the schools were ostensibly innovative; there very little comment on how they themselves became innovative. The publication of a framework for being innovative is something which can be of use to all schools but it lacks context. It lacks the narrative of how schools got to the that point, the mistakes they made, the route which they took to arrive their current position. A framework glosses over these by providing the ingredients without providing the recipe.

The literature on Innovation Studies appeared to be nothing but a similarity in phrasing on first reading. As a subject based on Design & Technology there appeared to be little in common. Upon further reading, it appeared to be using a fairly appropriate nomenclature as they had included quite a few innovative student skills within in. The focus on prototyping and solutions to real-world problems is definitely a form of innovation. They also touched on innovation as students skills, which was not seen a lot in other areas. Most of the school-based literature tended to focus on schools as whole; innovative teaching, including technology and innovating around other areas of school management. 



\cite{StopI4:online}