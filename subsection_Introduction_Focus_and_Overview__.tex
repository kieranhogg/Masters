\subsection{Introduction, Focus and Overview}

The United Arab Emirates is one of the fastest developing countries in the world\footnote{It has been on the World Bank list of high-income economies since 1987 \cite{worldbank} and is 41st in the UN Human Development Index \cite{WorkforHumanDevelopment2015}}, having only been established in 1971, which is developing much of its public services at a rapid rate to match the massive influx of people migrating to the country. Education is no exception to this with an increase in both non-local educators working in education, and workers in other sectors bringing their children to be educated in the country. This has lead to an education system which is both rapidly catching up with that of much older countries, and one which does not have the history and baggage of those countries. As a result, the education system can be much more mobile and can afford to be more forward thinking that more established systems. 

At the start of the 2015-2016 academic year, the inspection framework for schools in the UAE was published by the Abu Dhabi Education Council (ADEC). One of the new additions to the framework, which otherwise looks broadly similar to most developed countries', was the addition of innovation as an additional focus. To put it into perspective, it is given roughtly the same amount of space in the framework as more established educational ideas such as inclusion. ADEC discuss innovation as follows:

\begin{quote}
Innovation comes in many forms. There are innovations in the way schools are owned, organised and managed; in curriculum design models; in teaching and learning approaches, such as the ways in which learning technologies are used; classroom design including virtual spaces; assessment; timetabling; partnerships to promote effective learning and engagement in the economy; and the ways in which teachers and leaders are recruited, trained, developed and rewarded. These innovations can be small or large, recognisable or entirely new and different.

Innovation is driven by a commitment to excellence and continuous improvement. Innovation is based on curiosity, the willingness to take risks and to experiment to test assumptions. Innovation is based on questioning and challenging the status quo. It is also based on recognising opportunity and taking advantage of it. Being innovative is about looking beyond what we currently do well, identifying the great ideas of tomorrow and putting them into practice.
\end{quote} \cite{ADEC2015}



Broadly speaking, the initiative will be looked at in the following ways: 

\begin{itemize}
\item What is innovation in education and what is the goal of the initiative?
\item What form of innovation has taken place in schools so far?
\item How consistent is the interpretation of innovation between schools across the region?
\item How do the above answers compare to a more developed educational system?
\item Has there been any impact so far? What is the predicted future impact?

\end{itemize}

Hypothetically I think that there will have been some good work from schools so far but that how schools have interpreted it and implemented it so far will differ. That is not to say that this is a bad thing, one of the hopeful outcomes of this research is a clearer picture which will enable other educators to learn from others.

I think the difference in implementation will be a reflection of the diversity present in education in the UAE with various different types of educational establishments, staffed by many different nationalities, which are all at different stages of their educational lives.

---
\begin{itemize}

\item  What is the focus of your enquiry?
\item What questions are you trying to answer?
\item What are your hypotheses or hunches?
\item Why are these questions and hypotheses important to you and your institution?
\item What contextual information is important to bear in mind? (E.g. 'the school is single sex' or 'the children are year seven')

\end{itemize}