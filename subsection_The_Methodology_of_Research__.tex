\subsection{The Methodology of Research and Justification}
The overall purpose of the research is to the find the extent, if any, the focus on innovation in Abu Dhabi schools has changed schools. If schools have changed, what has changed and has there been any observable impact?

As previously discussed, the idea of innovation as a stand-alone concept, and more specifically, when placed in an educational context is not one that has an agreed definition. This leads to the obvious problem of asking people to discuss an area they themselves may not be completely clear on the definition. Rather than circumvent this issue by providing a context in which the research is conducted into, it has itself been included.

One of the areas for research is how well the idea has itself been understood, and to what level of uniformity has been achieved when developed innovation within their own schools. This could lead to more variance in the data received which make an assumption that the subject knows what innovation is, but the trade-off is made to get a clearer overall picture.

\subsubsection{The purpose of the research}
The research is best categorised as an investigation therefore the main method of research will be a pragmatic approach. The brief handed to schools has been fairly open and the length of time schools have had to implement it is short. The expectation of large culture changes is obviously unfounded however the level to which schools have acknowledged, understood and implemented any innovation initiatives is the overall focus. 

Taking a non-pragmatic approach would lead to such a narrow set of results and conclusions that any proposals made based on them would likely not apply to the wider community. 

It is important to note of the power dynamic in this ecosystem, one which is crucial to bear in mind when conducting the research, or indeed reading any results or conclusions. The innovation initiative is one that has been proposed and distributed by an external body to schools, a body which schools are otherwise held account to. Anyone participating in this research, especially anyone in a more senior leadership role, will therefore have a tendency to portray an image that perhaps be more generous that may actually be the case. To minimise this effect, the research will be conducted face-to-face where it is thought that a disclaimer or explanation would facilitate more honest dialog. Additionally, all quantative research conducted will be anonymous and where possible, independent of institution in order to gain the most accurate data feasible.

\subsubsection{The nature of the knowledge which is created}
\subsubsection{Types of research}
\subsubsection{The reason for your choice of research methodology}
\subsubsection{The possible measures for trusting the research (validity and reliability)}
\subsubsection{The nature and form of what may be generalised from the research.}

While the difficulty in getting accuracy in responses due the differing opinions on what innovation in education is, this will not diminish the validity of those responses, in themselves they are valuable data points. The generalisation of the research will be twofold: to what extent is innovation beneficial in schools and how best to achieve an innovative school; and reaction to a community-wide initiative and the level of success in doing so.

\subsubsection{The power relationships and involvement of stakeholders and the effects that this may have.}
\subsubsection{The ethical basis of the research.}
Ethically, the sole issue is anonymity. The relative level of success with the initiative will depends on a large number of factors, some which may not reflect positively on the staff or schools taking part. Individuals taking part have been guaranteed their anonymity as well as anonymising the schools taking part. With a range of schools and the nature of innovations such as this, there will inevitably be some results which look less favourable for some schools. It is also important to ground this research in one very specific area which is a small part of inspection framework which means no inferences can be made on the schools in the wider context.

This section should also:
•	Identify any bias that the researcher may bring to the research and how they intend to overcome this
•	Identify any variables and how they can will be dealt with
•	Outline how your research outcomes could be generalisable
•	The Tests for Truthfulness, Reliability and Validity that may be applied
•	The Involvement and Relationships between Researcher, and Subjects



