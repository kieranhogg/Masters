\section{Chapter 1}
\subsection{Introduction, Focus and Overview}
2,000 – 2,500 words\citep{Frame61:online}

Innovation is a word and concept most are familiar with; it is applicable to almost every industry. It is also not a buzzword, in Figure \ref{fig:popularity}, the popularity of innovation as a search term from the Google search engine is shown. Aside from the bi-annual dip in July and December (when industry quiets down for annual holidays), the popularity of the term has remained fairly constant since 2004, if anything it has become slightly less popular over time.

[h!]

In the 21st century, innovation for most people would conjure images of technology and the Internet, unsurprisingly, the most closely-related phrase according to Google is "technology innovation"\footnote{https://www.google.com/trends/explore\#q=innovation}. As an innovation, the increase in prevalence of home technology and access to the Internet is rightfully the most common example. A close second is "business innovation" which is where the word is commonly used, and where companies such as Apple, Google and Facebook who would be considered innovative, are combining the two areas of innovation to outstanding success.

One of the areas perhaps not as traditionally thought of as being innovative is education, at least not to the layperson. The idea of an educator who holds knowledge and dictates that to students has changed somewhat over the last hundred years but the core idea of that is still seen in most classrooms even today. For those more familiar with educational developments will appreciate that while this is true, there have been, and continue to be, many changes and innovations that take place within education. The three-part lesson has been introduced and remained popular; learning styles became popular and are now on the way out (TODO: ref); personalisation and a better understanding of SEN students has been a huge positive development for the better.

At any given time, there are numerous entities researching and implement different ideas and changes in education, some hugely innovative, some less so. Sometimes the impetus for change comes from research institutions, governments, sometimes from schools, sometimes from teachers or occasionally, from other agents such as parents and students. 

This research will focus on an aim to produce a more innovative education system in the United Arab Emirates, introduced by the UAE government.

The United Arab Emirates is one of the fastest developing countries in the world\footnote{It has been on the World Bank list of high-income economies since 1987 \cite{worldbank} and is 41st in the UN Human Development Index \cite{WorkforHumanDevelopment2015}}, having only been established in 1971, which is developing many of its public services at a rapid rate to match the massive influx of people migrating to the country. Education is no exception to this with an increase in both non-local educators, and workers in other sectors bringing their children to be educated in the country. This has led to an education system which is both rapidly catching up with that of much older countries, and one which does not have the history and baggage of those countries. TODO: add PISA here

The UAE comprises seven emirates. The largest and home to the nation's capital city is Abu Dhabi. Schools in Abu Dhabi come in two main forms, public schools offer free education to mainly Emirati children (expats must be under 20\% of the school, speak Arabic and pay tuition fees) and offer either an ADEC or Ministry of Education curriculum depending on the type of school. The other being private schools which are fee-paying and typically follow a foreign curriculum such as British or American. Whilst private schools are more independent in their operation and curriculum, ADEC still has a role in inspecting these schools to ensure overall quality and compliance across the emirate.

At the start of the 2015-2016 academic year, a unified inspection framework for schools in the UAE was published by the ADEC and their counterpart organisations from different emirates. One of the new additions to the framework, which otherwise looks broadly similar to that of most developed countries', was the addition of innovation as an additional focus. To put it into perspective, it is given roughly the same amount of weighting in the framework as more established educational ideas such as inclusion. ADEC discusses innovation as follows:

\begin{quote}
Innovation comes in many forms. There are innovations in the way schools are owned, organised and managed; in curriculum design models; in teaching and learning approaches, such as the ways in which learning technologies are used; classroom design including virtual spaces; assessment; timetabling; partnerships to promote effective learning and engagement in the economy; and the ways in which teachers and leaders are recruited, trained, developed and rewarded. These innovations can be small or large, recognisable or entirely new and different.

Innovation is driven by a commitment to excellence and continuous improvement. Innovation is based on curiosity, the willingness to take risks and to experiment to test assumptions. Innovation is based on questioning and challenging the status quo. It is also based on recognising opportunity and taking advantage of it. Being innovative is about looking beyond what we currently do well, identifying the great ideas of tomorrow and putting them into practice.
\end{quote} \cite[p.12]{ADEC2015}

The idea of innovation in education is not a new one, but the inclusion in such a prominent position of an inspection framework indicates the importance that ADEC and the UAE is placing on innovation, indeed they make reference to the UAE Vision 2021 as justification to the inclusion of innovation as an educational focus. There are many references to innovation in the vision, but the clearest one states: "We want the UAE to transform its economy into a model where growth is driven by knowledge and innovation." \cite{UAEGovernment2012} The importance placed in this area, the slightly surprising inclusion and the lack of exposure to it were the driving factors for choosing this as an area to focus on.

Due to the age of the initiative, the research will be conducted in an inductive manner; that is there will be no formal hypothesis to test. This research will follow the development of this formalisation of innovation in the academic year following its introduction. The aim is to both gauge the response from schools to the addition, but also to collate best practices for others to use.


Hypothetically I think that there will have been some good work from schools so far but that how schools have interpreted it and implemented it so far will differ. That is not to say that this is a bad thing; one of the hopeful outcomes of this research is a clearer picture which will enable other educators to learn from others.

I think the difference in implementation will be a reflection of the diversity present in education in the UAE with various different types of educational establishments, staffed by many different nationalities, which are all at different stages of their educational lives.

Broadly speaking, the initiative will be looked at in the following ways: 

\begin{itemize}
\item What is innovation in education and what is the goal of the initiative?
\item What form of innovation has taken place in schools so far?
\item How consistent is the interpretation of innovation between schools across the region?
\item How do the above answers compare to a more developed educational system?
\item Has there been any impact so far? What is the predicted future impact?
\end{itemize}

\section{Needs Analysis and Justification}

The need for this research varies from party to party. The common need for  all is to provide some clarity and context in the first instance. As a relatively new and novel (at least as an assessed outcome) focus, there is still some uncertainty surrounding it.

As a teacher within an Abu Dhabi school, there are a few different needs that may be satisfied by this research. With innovation seemingly playing a large part in the future educational landscape in Abu Dhabi and the UAE, as a teacher I would like to be as informed about it as possible. As there is not an immediate wealth of information regarding how I should go about it as a classroom practitioner, this will aim to fill that immediate need.

As part of a team, the need is not only to provide some best practices but to provide some context and evaluation of what we and other establishments are doing within the local area. There will be some examination and scrutiny over initiative that are already in place as to their effectiveness.

More broadly, the innovation focus has been seemingly added as a reaction to a deficit to improve the education system in certain areas. The UAE is aiming to bring its educational system up to the highest global standards as they have done so in other areas of society successfully. This is not done in a vacuum however, and this research will look at the origins, process and possible goals in creating an innovative educational system. While not common, it is not without comparison, so by analysing it with a global and historical perspective, any successes and challenges may be able to be inferred from other nations and other initiatives worldwide.


\subsection{Professional Autobiography}
My educational background has always been heavily focused around computers and technology. My secondary education took in A-Level ICT and Computing, which naturally led me to a degree in Computer Science. During my degree I was fortunate to work for a software company for a year full time and a further year part time. This was followed by a PGCE in ICT.

Upon my qualification, ICT qualifications taught in England were still very prescriptive and based around the teaching of Microsoft Office skills. As a result of my background, I felt naturally inclined to innovate with technology and the curriculum where it allowed. As a result, I believe that innovation is not the use of technology in the classroom by itself. I believe it can be a part when used correctly, but simple use of technology is not. To take an extreme example, effective use of interactive white boards is something mentioned within this sphere, indeed many schools will still offer new staff training on them. Perhaps surprisingly, white board were developed in 1990 and by 2007, 98\% \cite{kitchen2008harnessing} of secondary schools had them in every classroom. They are simply tools of teaching trade now. They do however, offer a positive example of a technological innovation improving teaching and learning, having become such a mainstay in modern classrooms.

My natural inclination therefore is to see innovation within education as holistic rather than one particular idea. This will obviously affect my initial opinion but will have no effect on the analysis of the data relating to how other teachers view it.

The use of technology has always come naturally so my inclination has always been to pursue technology in an effort to make my teaching and learning either easier, or more engaging. The reason for innovation and technology becoming almost synonymous in technology is twofold: a lot of technology can be genuinely innovative and can be "easy win" with its prevalence and impact, and secondly for many teachers, technology is the most obvious way to innovate within the classroom. I therefore tend to view innovation more favourably than most; I very very rarely have negative associations with it the way some teachers do, especially when it comes to technology.

My interest in the topic is comprised of a few different angles of personal relevance. I currently work in a new school which is not only innovative at present, but has many innovative plans for the future. I am keen to see innovation as a driving force for change in schools rather than as a tick box used during inspections. During my career I have had the opportunity to observe and work with many innovative projects, both technological and otherwise. Some have worked, some have not, as is the case for the innovation process. The projects based around technology have proved to be useful lessons from the perspective of someone with a great interest in the area. With the focus on innovation being so prominent, I am keen to be able to find similar examples in the local area to collate and share within this research.

Teaching is both an amazingly collaborative profession and an incredibly individual profession. The collaborative nature is evidenced by the amazing amount of resources online and the rich personal learning networks that have emerged on platforms such as twitter. In contrast, anyone who has written a lesson plan because an online one wouldn't be quite right will identify with teaching being inherently personal and individual  in nature as well. One of the hopes of this research is to share some best practice in this area.

To mitigate the potential personal bias for these areas, I will be ensuring the majority of data and conclusions are drawn from sources that are not me. To the greatest extent possible, questions will be written in a neutral and open way to prevent any leading questions.
As an employee in a school, I may be inclined to bias towards my institutions direction and practice. There must be a certain amount of critical thinking towards innovative practices however, but any side-by-side comparisons will be limited to good practice to ensure that these biases do not appear in the form of diminishing others’ achievements and our shortcomings. For the valid area of criticism of the idea and any practice stemming from it, that will be drawn out from staff at their institutions to ensure the analysis is as fair as can be achieved.
The research is also heavily based on Abu Dhabi education and to a slightly lesser extent, Dubai. Generally speaking, Dubai leads the way for education in the UAE, TODO: ref Abu Dhabi is shortly behind, with the five smaller and less wealthy Emirates (referred to at the northern emirates) some way behind. TODO: PISA Within the timescale of the research, it would not be possible to do a country-wide investigation. Therefore, some of the findings and conclusions may well not be applicable to the current timeline in the northern emirates, but it is hoped that even though it may not currently apply, many of the findings will be still be relevant given the innovation drive is a national one.

