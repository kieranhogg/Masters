\section{Chapter 2 - A Critical Review of the Literature}
4,000 – 5,000 words
One of the first challenges when approaching a review of the literature is to be able to define the scope of the search. In order to be able to accurately discover appropriate material, we must attempt to define innovation within the context of education.

\begin{quote}
The variety of ways in which the concept of innovation has been defined by researchers reflects the nature of the discipline (Gopalakrishnan & Damanpour, 1994), the level at which innovation is conceptualised, and whether it is being conceived as a product or process (Amabile, 1988; Kanter, 1988). According to one of the earlier definitions, innovation is an idea, practice or material artifact perceived to be new by the relevant unit of adoption (Zaltman et al., 1973). Later, Anderson and King (1993) conceptualised innovation as the emergence, import or imposition of new ideas which are pursued towards implementation, through interpersonal discussion and successive remoulding of the original proposal over time. This contemporary definition not only describes the nature of innovation, but also refers to the intrinsic process of implementation. With advances in research, the concept of innovation has also been refined and a more comprehensive understanding of innovation has emerged. It may be defined as the introduction and application within a group, organisation, or wider society, of processes, products or procedures new to the relevant unit of adoption and intended to benefit the group, individual or wider society (West & Farr, 1990).
\end{quote}  \cite{Sharma_2005}

\cite{StopI4:online}