\section{Analysis}
5,000 words
\begin{quote}
Present here your results with a full discussion and interpretation. You need to analyse what your findings mean within your work context and fully interpret them, comparing them to the literature you have reviewed and other findings in your research.

Where you are presenting a lot of numerical data, it is important to include a visual representation. You can display data in, for example:
•	Tables
•	Graphs (bar, pie, scatter etc..)
•	Charts 
•	Diagrams
•	Vignettes

Immerse yourself in the data and becoming familiar with it
Code it 	identifying categories to place data into.
Themes
Patterns
Hypothesis testing
Linking- Connecting
Combine data sets: i.e. comparing observation, questioning and documentary data
Weighing: Discuss
•	Measuring/evaluating the weight of the evidence
•	Saying where there is strong and weak evidence, or none at all. 
•	Sources of bias
•	Descriptive Validity
•	Explanatory Validity
•	Reliability of the results

Link your analysis to the literature review and knowledge that has already been created. This is an important chapter in which you make sense of your results and is likely to be the longest. Ensure that you clearly cross reference to your Appendices, where appropriate, and do not just list where the evidence can be found.

Evaluate the reliability and validity of your evidence. How strong is your evidence?  How far we can rely upon your research to base our practice or are there inherent flaws? You might answer the questions: -

•	How strong is your evidence for each of your findings? (E.g. strong, some or weak evidence) 
•	How good were your research tools?
•	What are the sources of bias in your results?
•	How far is your work a valid description? (I.e. descriptive validity)
•	How secure is your analysis? 
•	If the research were done again would it produce the same results? (Reliability)
•	What difference has this research made to you as a professional and a person?
•	What difference has it made to your institution?
•	What do you intend to do next as a consequence of your research
\end{quote}

\subsection{Introduction of Innovation}
At the time of writing, less than one academic year has passed since the introduction of the new framework which highlighted innovation. Starting at the beginning, we look at the data provided from the questionnaires from teachers within the emirate. Of those who responded, 75% of replies indicated that yes, their school had made changed since the start of the academic year. 12.5% replied they had not, 12.5% were unsure. Given the varying nature of the job responsibilities for those who responded, the results are indicating that most schools have acknowledge and embraced the changes.

Of those who has replied that their school had made changes, they were invited to provide examples of the changes made. Example changes included:

\begin{itemize}
\item Subject specific innovation classes. Encouraging and demonstrating innovative teaching tools and methods
\item Innovation day (all school project), teaching with I-pad, different apps, Flipping the classroom, use of augmented reality, online assessment, etc....
\item New SEN programmes have been trialled.
\end{itemize}

One respondent replied with some examples from their inspection report which have been edited for relevancy:

\begin{itemize}
\item Students demonstrate excellent communication skills, they use these very effectively to share ideas and explain their thinking
\item Students collaborate extremely effectively
\item They can work effectively in pairs or small groups, listen attentively to each other, negotiate their responses in older year levels and make creative and thoughtful presentations
\item Children in KG are empowered in a well-structured environment to make choices and develop secure independent skills and creativity in a range of situations
\item Students in all years have acquired confident use of iPads and other digital technologies and use them in all subject areas
\item Students have a very strong work ethic when working independently and within groups, highlighted in collaborative participation in [a project], responding to STEM challenges
\item Most teachers provide opportunities for students to problem solve and respond to probing questions in planning their lessons
\item Students routinely respond confidently to ‘why’, ‘can you explain’, ‘how’ questions
\item By Years 12 and 13, students confidently ask challenging questions of each other
\item Curriculum review and modification leads to extensive opportunities for students in all years to innovate, enhance their learning and show enterprise. For example, using iPads within the continuous provision in FS and Years 1 and 2; the accelerated reader programme in primary and secondary; and Year 9 managing the complex process of publishing a high quality book of student writing
\item Senior leaders have embedded a culture of innovation into curriculum planning and development
\item Staff embrace opportunities and appreciate the support and encouragement to promote innovation within teaching and learning
\end{itemize}


While there are undoubtedly some great examples in here, it is also illuminating to note that both group work and effective questioning are highlighted as good examples of innovation. By any definition of innovation that is quite a stretch, both ideas having been cemented in good teaching practice for many years. It is an important thing to note as it does backup the theory that innovation is quite hard to define and measure, as seen by the highlighting of everyday good teaching practice as innovation by a group of experienced inspectors.

\subsection{Reaction to Introduction}

\subsection{Forms of Innovation}

\subsection{Case Studies}
During the process of interviews and questionnaires, based on the early feedback that was received, three schools were chosen to be looked at in more depth. These three schools would be categorised as above average in innovation. They are not necessarily the three most innovative schools in the area, even if that could be measured empirically, however all three have some excellent forms of innovation which made them stand out.

\subsubsection{School A}
School A is a premium\footnote{As of 2014, no later information is available for the categories https://www.adec.ac.ae/en/MediaCenter/Publications/PVT\%20Schools\%20end\%20of\%20year\%20report-\%20Irtiqaa\%20eng/HTML/files/assets/basic-html/index.html#40} adec British Curriculum School in Abu Dhabi which is relatively new school. Its inspection report was completed prior the this academic year, resulting in no reference to innovation within it.

Whilst not directly innovation, the school focuses on hiring the best teachers available, with a rigerous hiring process which is still conducted directly by the headmaster. The impact of this on innovation is hoped that by hiring the best teachers, the level of teacher will be such that they are naturally innovative in their practice.

As with many schools, they have introduced specialised innovation weeks where students are off timetable and perform activities which are outside the usual curriculum, whether that be the activity itself, the delivery or the method of learning. The younger students participcated in an innovation week which focused around Computing and Science with students taking part in more investigative approach to learning with more student-led opportunitues, more groupwork and less structure in terms of the timetable. The older years have partipated in a innovation week where mostly STEM subjects took part.




\subsection{Opinions on Innovation}


\subsection{Other Themes}
* Polarising opinion on technology

\subsection{Best Practises/Case Studies}

\subsection{Future of Innovation}


What does this mean for:
\begin{itemize}
\item me
\item my school
\item other schools
\item the area/ADEC
\item worldwide
\end{itemize}

\subsection{SCF}
* ADEC coming in 'inspect' innovation - leads to obvious