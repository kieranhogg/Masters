\section{Chapter 4}
\subsection{Project Activity, the Processes of Conducting Research}
2,000 words 

During the design stage of the questionnaires, there were several prototype questionnaires designed which were gradually improved on based on feedback. When in the midst of research, it is very easy to make assumptions about the knowledge of the respondents. The first prototypes were met with a fair amount of confusion as while teachers had vaguely heard about the innovation element of the framework, they were unaware that this what the focus of the research. This resulted in an addition of an explanatory paragraph which described the context in which the research sat. Once this change was made, respondents felt more confident in responding.

Some of the questions were found to be too direct and too leading. Initial prototypes included a question relating to the 'final form' of innovation, that is, what did the respondent see as the ultimate outcome for innovation, as a student skill, teaching style, school attitude etc.. The feedback regarding this question was that many did feel in a good position to answer as it was not in their realm of expertise. The goal of the question was to investigate which of the strands of innovation people most identified with. Subsequently it was decided to investigate this area in a less obvious manner and analyse the responses for the questions related to innovation definition, alongside what schools are currently doing.

The response rate was lower than even a pessimistic estimate with a target of around 100, only 18 were received back. It was previously mentioned that the aim of the questionnaires was not to be statistically significant, but such a low number made it more difficult to select quality, varied candidates for the schools to investigate further.

The senior staff interviews proved to be as useful as had hoped and the data collected, while qualitative, gave for some useful starting points. They also addressed misconceptions and laid the foundation for the rest of the research. The data from the in-person interviews was the most difficult to collate and analyse but perhaps the richest form of data collected during the process. As expected, the challenges of finding convenient times in which to interview senior leaders proved to the greatest challenge. This was the main driving force for the use of questionnaires given their ability to case a wider net. 

An unforeseen factor in the interviews was the period of time in which the person was interviewed. Early in the research process, the interviews were more structured, in a similar form to the questionnaires only with more scope for follow-up questions. As the research process developed, so did the interviews. The interviews became less structured, more exploratory and more forward-thinking.

Despite this, both styles were ultimately useful and the order in which they were performed proved likewise, the interview process ended up more evolutionary than structured. Access to senior leaders did prove difficult at times, those who were accessible were valuable but the amount of leaders willing to discuss the innovation was not as desired. This was due in part to their schedules, there was not a lot to gain from participating in this research, even schools doing great work are anonymous so there is not even a self-publicity element. The other aspect discussed previously was that this is a government-led initiative. Those schools who are not already achieving at a high level, will naturally have given this focus a lot less emphasis than more obvious areas in teaching and learning. 

The observations conducted by observers in schools resulted in rich, but unstructured data. As the observational log was written over a relatively long period of time, projects and ideas that stretched throughout the year were written separated by other smaller items that occurred during. The data was extremely valuable but it would have been better if it was not structured chronologically and was arranged into areas or projects to make it easier to interpret.


