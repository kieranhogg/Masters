\subsection{Professional Autobiography}

\begin{quote}
This section is about articulating your experiences, beliefs and values on the topic of your choice. This might help to focus your study and will enable you to identify and explore your personal bias in the account. 

A “professional autobiography” in research terms is seen by some as fundamental. The argument is made that your experiences have influenced your beliefs and values and how you see the world. You life and career experience has led you to ask the questions that you are asking and will influence your analysis of them.

This argument suggests that the inherent bias of the researcher is best exposed and acknowledged rather than claiming that your research is completely objective. 

Produce an autobiography that reflects you as you have developed to become the professional you are - include: 
\begin{itemize}
\item Key moments or critical incidents that have shaped the way you think and work, 
\item People, research, experiences and or theories that have been important to you in your career and the influence that they have had on you and why;
\item The issues and concerns you have;
\item Why you are interested in your current topic;
\item Outline any bias you may bring to the research and how you intend to deal with these;
\item What beliefs and values are revealed by these?
\end{itemize}
\end{quote}
My educational background has always been heavily focused around computers and technology. My secondary education took in A-Level ICT and Computing, which naturally led me to a degree in Computer Science. During my degree I was fortunate to work for a software company for a year full time and a further year part time. This was followed by a PGCE in ICT.

Upon my qualification, English ICT qualifications were still very prescriptive and based around the teaching of Microsoft Office skills. As a result of my background, I felt naturally inclined to innovate with technology and the curriculum where it allowed. As a result, I feel fairly strongly that innovation is not the use of technology in the classroom. I believe it can be a part when used correctly, but simple use of technology is not. To take an extreme example, effective use of interactive white boards is something mentioned within this sphere, but they were developed in 1990 and by 2007, 98\% \cite{kitchen2008harnessing} of secondary schools had them in every classroom.

My natural inclination therefore is to see innovation within education as holistic rather that one particular idea. This will obviously affect my hypothesis but will have no effect on the analysis of the data relating to how other teachers' view it.