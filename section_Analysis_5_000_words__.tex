\section{Analysis}
5,000 words
\begin{quote}
Present here your results with a full discussion and interpretation. You need to analyse what your findings mean within your work context and fully interpret them, comparing them to the literature you have reviewed and other findings in your research.

Where you are presenting a lot of numerical data, it is important to include a visual representation. You can display data in, for example:
•	Tables
•	Graphs (bar, pie, scatter etc..)
•	Charts 
•	Diagrams
•	Vignettes

Immerse yourself in the data and becoming familiar with it
Code it 	identifying categories to place data into.
Themes
Patterns
Hypothesis testing
Linking- Connecting
Combine data sets: i.e. comparing observation, questioning and documentary data
Weighing: Discuss
•	Measuring/evaluating the weight of the evidence
•	Saying where there is strong and weak evidence, or none at all. 
•	Sources of bias
•	Descriptive Validity
•	Explanatory Validity
•	Reliability of the results

Link your analysis to the literature review and knowledge that has already been created. This is an important chapter in which you make sense of your results and is likely to be the longest. Ensure that you clearly cross reference to your Appendices, where appropriate, and do not just list where the evidence can be found.

Evaluate the reliability and validity of your evidence. How strong is your evidence?  How far we can rely upon your research to base our practice or are there inherent flaws? You might answer the questions: -

•	How strong is your evidence for each of your findings? (E.g. strong, some or weak evidence) 
•	How good were your research tools?
•	What are the sources of bias in your results?
•	How far is your work a valid description? (I.e. descriptive validity)
•	How secure is your analysis? 
•	If the research were done again would it produce the same results? (Reliability)
•	What difference has this research made to you as a professional and a person?
•	What difference has it made to your institution?
•	What do you intend to do next as a consequence of your research
\end{quote}

\subsection{Introduction of Innovation}

\subsection{Reaction to Introduction}

\subsection{Forms of Innovation}

\subsection{Other Themes}
* Polarising opinion on technology

\subsection{Best Practises}

\subsection{Future of Innovation}


What does this mean for:
\begin{itemize}
\item me
\item my school
\item other schools
\item the area/ADEC
\item worldwide
\end{itemize}