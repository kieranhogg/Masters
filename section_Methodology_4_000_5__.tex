\section{Methodology}
4,000 - 5,000 words
\begin{quote}
This section should contain your plan for the collection of data which should be systematic and realistic, taking into account any variables or problems you may come across. It should also include how you are going to triangulate your findings.

Discuss the tools you will use to collect your data covering the following:

Collecting data using observation
Will observation be a strategy you will choose to use as part of your data collection process?
If so:
1.	Will you be the best person to undertake the observation or is it that you want to be observed?  If the latter, who could you ask to be the observer?
2.	Who will you choose to observe?
3.	How will you select people to be observed?
4.	What will you be observing?  Try to be specific.
5.	Do you need the co-operation of other people in order to observe?  For example, another class teacher.
6.	When and where will the observations take place?
7.	How long will you observe for?
8.	How will you record your observations?
9.	Who has access to your observations?
10.	Have you had to produce your own tool or are you using a published one?
11.	Have you had a trial run and tinkered with the schedule

Collecting data using interviews
1.	How many interviews will you decide to conduct?
2.	Who will you want to interview and why?
3.	How will you decide whom to interview?
4.	What kind of interview will you conduct - e.g. a formal interview with set questions, a more informal interview with some questions but space for the interviewee to give you a more personal reflection or a completely unstructured interview around a given topic
5.	How will you record your interviews?
6.	How will you need to transcribe the interviews?
7.	How you will interpret your interviews
	
Collecting data using questionnaires
1.	What questions do you need answering for your research?
2.	What types of questionnaire design are there?
3.	Why have you selected the method you have chosen?
4.	How do you know that you are asking the questions in a way that will elicit the types of response needed?  For example, are you going to ask people to tick boxes, give written responses, answer on a scale, etc.?
5.	Who will you ask to complete your questionnaires?
6.	Are respondents going to be identifiable or will the returns be anonymous?
7.	How will you collate the responses?
8.	How many questionnaires will you need to give out and receive back to make a viable sample?
9.	How will you select the sample.


Artefacts etc..
Outline any artefacts that will help you tell the story of your focus and how you will analyse and use them. State why you have selected them. These may include: -
	Learning materials
	Photographs
	Schemes of work
	School policies
	Children's work (written or otherwise)
	Minutes of meetings
	Letters from parents
	Ofsted reports
	School review reports
	Outcomes from performance management processes
	School Plan
\end{quote}