\section{Chapter 2 - A Critical Review of the Literature}
4,000 – 5,000 words

You must do more than list material, catalogue authors and their views, and much more than just describe what you have read. 

The aim of a literature review is to show your reader (your tutor) that you have read, and have a good grasp of, the main published work concerning a particular topic or question in your field. This should be an outline of what is known about your area of study.  The review should be the foundation for informing and justifying what you are going to do and will underpin and link to your findings in the research section.

It is very important to note that your review should not be simply a description of what others have published in the form of a set of summaries, but should take the form of a critical discussion, showing insight and an awareness of differing arguments, theories and approaches. It should be a synthesis and analysis of the relevant published work, linked at all times to your own purpose and rationale.

\textbf{A good literature review, therefore, is critical of what has been written, identifies areas of controversy, raises questions and identifies areas which need further research. }

It should start with an abstract along the following lines:
In this literature review I intend to look at …………….. and will attempt to outline………..

In the review:
\begin{itemize}
\item{Show knowledge of policy and key authors in the field}
\item{Analyse key ideas and key concepts. (What Definitions are there? What different meanings are there?)}
\item{Look for Strengths, Weaknesses, Opportunities, Threats}
\item{Undertake Gap Analysis}
\item{Compare and Contrast. }
\item{Identify and analyse }
\item{Compare and contrast.}
\end{itemize}
Evaluate:
\begin{itemize}
\item{Does it make sense? }
\item{How does it apply to practice?}
\item{How important is it? (To you, to learners, to examiners)}
\item{Your review must be written in a formal, academic style. Keep your writing clear and concise, avoiding colloquialisms and personal language. You should always aim to be objective and respectful of others' opinions; this is not the place for emotive language or strong personal opinions. }
\item{If you thought something was rubbish, use words such as "inconsistent", "lacking in certain areas" or "based on false assumptions"! }
\item{When introducing someone's opinion, don't use "says", but instead an appropriate verb which more accurately reflects this viewpoint, such as "argues", "claims" or "states". Use the present tense for general opinions and theories, or the past when referring to specific research or experiments}
\end{itemize}
Linking words are important. If you are grouping together writers with similar opinions, you would use words or phrases such as: similarly, in addition, also, again

More importantly, if there is disagreement, you need to indicate clearly that you are aware of this by the use of linkers such as: however, on the other hand, conversely, nevertheless

And remember at all times to avoid plagiarising your sources. Always separate your source opinions from your own hypothesis. Make sure you consistently reference the literature you are referring to. Put all quotes in italics between “ …” speech marks and acknowledge the source, giving date and page number.