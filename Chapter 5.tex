\section{Chapter 5}
\subsection{Analysis of findings and outcomes}
5,000 words
With the pragmatic nature of the research, the aim was not to prove or disprove any particular hypothesis, it was to investigate the entire context surrounding innovation in Abu Dhabi schools. With that said, there were several strands which were apparent even before starting the research: the relative surprise at the inclusion, and particularly, speed of inclusion; the introduction of innovation days or weeks many schools introduced, and what exactly constituted innovation.

During the research, these strands of research opened up further areas for investigation which were followed; some proved fruitful, others were dead ends and were not continued. There were some issues which only affected certain stakeholders, but where possible, data and analysis on topics which are generally similar in scope were brought together.

The aim is not to decide whether the innovation drive was a success or failure, even if that were empirically possible, there is not much to be gained from such finality only a year into the project. What it will aim to do is to look at innovation as an entity in itself; is it a worthwhile addition to a school inspection? The other aspect is looking at the addition of the innovation as a project, how was that performed, how has it gone and what, if any, lessons can be learned from that.

\subsection{Understanding of Innovation}
\subsubsection{Defining Innovation}
As previously touched upon, the term and concept of innovation was one which at times could be fairly loosely defined. One of the first questions posed was to ask them what their definition of innovation was within a school context. With one exceptions where both were mentioned, teachers' definitions of innovations fell into two categories: those related to teaching and learning and those that were more general. A sample have been chosen for illustration.

Related to teaching and learning: 

\begin{itemize}
\item "Incorporating current real life events, technological, cultural and political advancements into teaching and learning"
\item "Thinking differently about the way you teach / the way the students learn"
\item "Innovation in education is about constantly reflecting on our daily practice and our pedagogical approach, […] action research and looking on what we can improve and change in our teaching"
\item "Allowing students to face challenges is also allowing them to be critical thinkers and problem solvers".
\end{itemize}

Unrelated to teaching and learning:
\begin{itemize}
\item "A change in the value system within education. […] Understanding present context; researching, experimenting, analysing the new before we ring out the old. [...]"
\item "Innovation is being creative"
\item "Dare to try something new and not always guaranteed"
\item "Innovation is creativity and willingness to take risks in order to bring fresh, new ideas."
\item "[…] On a larger scale innovation comes from a wide range of components in schools' namely timings of the day, timetabling, lesson structure and opportunities."
\end{itemize}

Interestingly, the nature of the response had very little relation with the job title, with a mixture of teachers, middle and senior leaders providing answers from both categories. There is obviously some variation in the definitions of innovation within education, which was expected. Within the two categories however, definitions tend to have the same sentiment. The first category could be summed up by reflective and varying teaching practice, the latter by risk taking. Indeed, some of the definitions could equally be applied to teaching so the overlap is understandable.

An empirical comparison with the literature is not possible, but comparing the teacher-provided definitions with those identified in the literature, it is clear that the sentiments expressed behind the definitions are the same. Given most teachers will not have read the inspection framework, we can conclude that there is a reasonably well-held definition of innovation within education.

The framework provides a clear, definition for innovation: \begin{quote}
Innovation is the generation of new and creative ideas and the use of new or improved approaches. 
\end{quote}TODO: REF 

It further clarifies the driving force behind it: 
\begin{quote}
Innovation is driven by a commitment to excellence and continuous improvement. Innovation is based on curiosity, the willingness to take risks and to experiment to test assumptions. Innovation is based on questioning and challenging the status quo. It is also based on recognising opportunity and taking advantage of it. Being innovative is about looking beyond what we currently do well, identifying the great ideas of tomorrow and putting them into practice.
\end{quote}

A slight ambiguity is whether this refers to schools or students out of context, but this is clarified in the next chapter: it is referring to schools.


\subsubsection{Examples of Innnovation}

Having defined innovation, we progress to what innovation looks like, or at least, is thought to look like from a teacher’s perspective and from the framework's.

Some of the feedback arising from interviews was that the new inspection framework lacks guidance for practical examples of innovation. It could be argued that giving concrete examples of innovation would lead to a narrow implementation, but the consensus was some guidance would have helped schools gain confidence in their implementation.

From the framework firstly, innovation is included in students learning skills. The following two excerpts are referencing those skills. The brief descriptor for Outstanding student skills: 

\begin{quote}
Students are innovative and enterprising. They are independent learners and can find things out for themselves using a variety of different sources. They use learning technologies independently and very effectively. Critical thinking and problem- solving skills are intrinsic features of learning.
\end{quote}

The more detailed example:
\begin{quote}
The quality of learning skills illustrated below would be evaluated as outstanding.
\begin{itemize}
\item Students are motivated and eager participants in their learning. They are actively involved in their own learning and development, and show increasing skills as learners and assessors of their own learning. Students are very aware of their progress and strengths in learning. The questions they ask show they are making important connections between new learning and what they already know. They are reflective and analyse learning situations in order to discover the best solutions. Their independence shows itself particularly in the ways they use technology.
\item Students choose the best ways to complete tasks within group and individual settings both in leading and supporting their peers. Through effective collaboration with others, by contributing ideas and listening to one another, students demonstrate high levels of skills as independent thinkers and learners, and achieve common goals.
\item Skills, knowledge and understanding acquired are applied confidently and accurately to new learning contexts. Students demonstrate success in applying their skills to problems reflecting real life situations, both familiar and unfamiliar. They make connections between their learning in different parts of the curriculum. They are successful, confident, responsible learners.
\item Students demonstrate proficiency in finding out new information and are able to apply successfully their critical thinking to tasks. They are innovative and creative. They hypothesise and draw inferences with ease and so their abilities to solve problems are excellent. Their work will often reflect maturity and independence of thought and they readily find things out for themselves by using books and other resources, including technology. Through the effective use of different sources of information, students are able to make accurate and appropriate conclusions and present their learning with confidence
\end{itemize}
\end{quote}

Taking the brief descriptor, the first two sentences cover working independently and excelling with ICT, while valuable skills, educators would probably not see anything innovative about those two. This leaves us with critical-thinking and problem-solving skills, this seems to align more closely with the teacher definitions. These are still fairly vague, so let analyse the broader description. There are some more useful aspects to draw out from these sections, but the following quote still shows the vagueness of innovation: "They are innovative and creative". Compare and contrast that statement with a surround one such as: "Through the effective use of different sources of information, students are able to make accurate and appropriate conclusions and present their learning with confidence". There is very little room for ambiguity in much of the framework, as one would hope when being evaluated using it. The lack of definition could be looked upon in two ways: is is left entirely up to schools as a freedom to implement, while freeing, this makes any empirical judgment difficult, the second being that ADEC (or at least the authors) are not completely sure of what it would look like.

Using the data collected from staff and students, the closest thing to a definition included in the framework would be: "Skills, knowledge and understanding acquired are applied confidently and accurately to new learning contexts. Students demonstrate success in applying their skills to problems reflecting real life situations, both familiar and unfamiliar. They make connections between their learning in different parts of the curriculum." This incorporates the problem-solving and creative-thinking aspect as well as matching "thinking outside the box" from another ADEC publication which will be discussed in a later section. The inclusion of innovation \textit{after} this implies this may not the accepted definition.

In light of this ambiguity in the inspection framework, the next obvious place to look is at an inspection report. Any reports completed this year must have mentioned their evidence in the report. Looking at two inspection reports, one with outstanding innovation, one with areas for improvement, the following were mentioned specifically as good or suggested examples of innovation:
\begin{itemize}
\item Action research by teachers in partnership with Universities
\item Students supporting students, particularly with older students taking an academic mentoring role
\item Effective use of digital technologies
\item Student questioning opportunities, particularly when working on individual and collaborative projects
\item Active and purposeful enquiry 
\item Investigation, critical thinking and collaborative learning
\item [Conversely,] too much teacher direction restricts children’s ability to develop independence and innovation skills.
\item providing students with as many opportunities as possible to develop higher-order thinking skills, solve problems, carry out research, and learn independently
\end{itemize}

Here we have concrete examples of innovation given by school inspectors. All of them focus on teaching and learning, where any mentioned of innovation outside the classroom are changes to support innovation, e.g. cross-department planning, whole-school approaches. While useful to have concrete examples of innovation, looking back to the areas of innovation mentioned, this only addresses one of the areas from both ADEC, and from Tubin et al.'s framework TODO: do I need to ref? for innovative schools.

Further to the innovation definition, teachers were asked to give some good examples of innovation within schools. Those which were reasonably practicable for most have been extracted: 

\begin{itemize}
\item Concept based learning
\item A greater involvement of students in the teaching and learning process
\item A flattening of the hierarchical power structure within schools
\item Staff being up to date with current research, and current technological advances
\item Time and space for staff to pursue their own professional development
\item Radical collegiality - all staff and students working together to improve the learning process
\item Project Based Learning which has clear and effective cross-curricular links [...]
\item Non-traditional teaching methods that are effective, motivating and ensuring progress.
\item Real-world problems, problem solving and independent learning
\item Holistic approaches to pupil welfare.
\item The continual experimentation with various software and online learning resources aiding all students' learning is essential for creativity and innovation
\item Up to date CPD is essential on various new teaching methods and styles to develop teachers and students alike
\item The setup of classrooms / learning space.
\item Cross-curricular projects
\item Allowing students to transfer knowledge from one subject to another to make it relevant to them
\item Using technology in learning to suit students unique learning methods
\item Using 'Flipped Learning' to support and extend leaning
\item Team work, using stories in teaching [...]
\item Not being driven by fear, [...] the biggest blockage to innovation; being controlled and restricted by policy and fear of the outcome of creativity.
\end{itemize}

There are inevitably some areas raised in the list wherein one could make an argument for whether they were truly innovative or just good teaching practice. That said, the range and quality of real examples of innovation was perhaps the biggest surprise to arise from the data. Innovation was considered a difficult term to define. That has been shown by the use of the term in the inspection framework, as well as the variety of definitions, even if they did fit mostly into two categories. 

An interesting part of the examples is the amount of contexts they cover from the original ADEC proposal. Whereas the definitions focused on teaching and learning, there are far more high level and whole school contexts mentioned. There is an indication therefore, that despite innovation being hard to define, teachers know it when they see it.

\subsubsection{Students}

\subsection{Introduction of Innovation}
At the time of writing, less than one academic year has passed since the introduction of the new framework which highlighted innovation. Starting at the beginning, we look at the data provided from the questionnaires from teachers within the emirate. Of those who responded, 75\% of replies indicated that yes, their school had made changed since the start of the academic year. 12.5% replied they had not, 12.5\% were unsure. Given the varying nature of the job responsibilities for those who responded, the results are indicating that most schools have acknowledge and embraced the changes.

Of those who has replied that their school had made changes, they were invited to provide examples of the changes made. Example changes included:

\begin{itemize}
\item Subject specific innovation classes. Encouraging and demonstrating innovative teaching tools and methods
\item Innovation day (all school project), teaching with I-pad, different apps, Flipping the classroom, use of augmented reality, online assessment, etc....
\item New SEN programs have been trialed.
\end{itemize}

One respondent replied with some examples from their inspection report which have been edited for relevancy:

\begin{itemize}
\item Students demonstrate excellent communication skills, they use these very effectively to share ideas and explain their thinking
\item Students collaborate extremely effectively
\item They can work effectively in pairs or small groups, listen attentively to each other, negotiate their responses in older year levels and make creative and thoughtful presentations
\item Children in KG are empowered in a well-structured environment to make choices and develop secure independent skills and creativity in a range of situations
\item Students in all years have acquired confident use of iPads and other digital technologies and use them in all subject areas
\item Students have a very strong work ethic when working independently and within groups, highlighted in collaborative participation in [a project], responding to STEM challenges
\item Most teachers provide opportunities for students to problem solve and respond to probing questions in planning their lessons
\item Students routinely respond confidently to ‘why’, ‘can you explain’, ‘how’ questions
\item By Years 12 and 13, students confidently ask challenging questions of each other
\item Curriculum review and modification leads to extensive opportunities for students in all years to innovate, enhance their learning and show enterprise. For example, using iPads within the continuous provision in FS and Years 1 and 2; the accelerated reader programme in primary and secondary; and Year 9 managing the complex process of publishing a high quality book of student writing
\item Senior leaders have embedded a culture of innovation into curriculum planning and development
\item Staff embrace opportunities and appreciate the support and encouragement to promote innovation within teaching and learning
\end{itemize}


While there are undoubtedly some great examples in here, it is also illuminating to note that both group work and effective questioning are highlighted as good examples of innovation. By any definition of innovation that is quite a stretch, both ideas having been cemented in good teaching practice for many years. It is an important thing to note as it does backup the theory that innovation is quite hard to define and measure, as seen by the highlighting of everyday good teaching practice as innovation by a group of experienced inspectors.

\subsection{Reaction to Introduction}

\subsection{Students}
TODO: References School A too early
While talking of innovation, it is quite easily to get lost in discussions of innovative teaching and innovative schools. The ultimate aim of any good school of course, is to improve the quality of education for students. It can't be said that innovation is for this end goal, some innovations are for improving the work load of teachers or the workings of schools, but a good proportion will benefit students either directly or indirectly.

At School A, where innovation weeks were used, students opinions on these weeks were solicited. The two main focii were whether students enjoyed the sessions more than usual, and whether they learnt more. The responses are shown in \cite{enjoy} and \cite{learn}

\subsection{Forms of Innovation}

The confusion over definitions and examples of innovation were clear from interviews and observations, but the results to questionnaires were contradictory; it seemed most could defined and give examples.
When this was posed again to interview subjects, the different uses of the word emerged as the source of the confusion.
Innovation in education could be:
\begin{enumerate}
\item Schools are being innovation in their operation, planning etc.
\item Teachers are using innovative teaching and learning strategies 
\item Students are being taught to be innovative learners (innovation skills)
\item Introducing a new idea or project and the management of that
\item Creating an innovative education system over a wide area, e.g. Abu Dhabi/UAE
\end{enumerate}
None of these meaning contradict any definitions previously encountered and they could all come under the definition of innovation. The confusion seems to come from where the focus is, should be, and is expected to be. 
From the inspection framework, it seems the overall goal is point five, and from inspection reports, the focus is point two with a small discussion of point one.
This multitude of facets of innovation has no doubt caused confusion within schools initially, but even over the first year, as it becomes clear where inspection teams’ and school leaders’ priorities lie, the confusion is become less.


\subsection{Case Studies}
During the process of interviews and questionnaires, based on the early feedback that was received, three schools were chosen to be looked at in more depth. These three schools would be categorised as above average in innovation. They are not necessarily the three most innovative schools in the area, even if that could be measured empirically, however all three have some excellent forms of innovation which made them stand out.

\subsubsection{School A}
School A is a premium\footnote{As of 2014, no later information is available for the categories https://www.adec.ac.ae/en/MediaCenter/Publications/PVT\%20Schools\%20end\%20of\%20year\%20report-\%20Irtiqaa\%20eng/HTML/files/assets/basic-html/index.html\#40} British Curriculum School in Abu Dhabi which is relatively new. Its most current inspection report was completed prior to the this academic year, resulting in no reference to innovation within it.

TODO: Bollocks?
Whilst not directly innovation, the school is not in a, with a rigorous hiring process which is still conducted directly by the headmaster. The impact of this on innovation is hoped that by hiring the best teachers, the level of teacher will be such that they are naturally innovative in their practice.

As with many schools, they have introduced specialised innovation weeks where students are off timetable and perform activities which are outside the usual curriculum, whether that be the activity itself, the delivery or the method of learning. The younger students participated in an innovation week which focused around Computing and Science with students taking part in more investigative approach to learning with more student-led opportunities, more group work and less structure in terms of the timetable. The older years have participated in an innovation week where mostly STEM subjects took part.

The school is modeled after the UK preparatory (prep) school system with a prep and senior school. There are other such schools in the UAE, although relatively few in number. Like other schools they have a relationship with a UK-based school, albeit with a much stronger partnership that is normally seen. The relationship has already seen some excellent collaborative work with both staff and students and  more is planned. The use of overseas links to other facilities is more common in the higher education system with universities such as NYU, Heriot-Watt, Middlesex and Paris-Sorbonne universities opening satellite campuses in the UAE with close links to their parent campus. Until now, similar arrangements from schools have been more about branding, whereas this school is started to use the relationship more like the higher education establishments.

In terms of innovative teaching and learning, the school is going to use the Harkness Method\footnote{http://www.exeter.edu/admissions/109\_1220.aspx} within its sixth form lessons. Harkness itself cannot be considered innovative having been developed in the 1920s, but the original development of Harkness occured "when a philanthropist, Edward Harkness, approached the principal of Phillips Exeter Academy (PEA), offering to fund an innovative method of education that would improve American education." \cite{Sevigny2016} Its popularity has been increasing in recent years, particularly in private and independent schools; the method works best with small class s sizes. Referring back to our definitions of innovation which included problem solving and critical thinking, Harkness is "an approach to education that inculcates a culture of enquiry, driven by students in dialogue around a table. [...] The teacher is required to be more open-minded and less controlling over outcomes, to take the risk of listening more and saying less. [...] it is a useful symbol for a community committed to student discourse and problem solving." \cite{Williams2010} The Harkness method is closely related to the Oxbridge model of dialogical tutorials, a key factor in introduction them being: to better prepare students for University. It also fulfills aspects of innovation as set out by ADEC, permeating curriculum design, instruction and classroom layout. 

\subsubsection{School B}
School B is a premium International Baccalaureate (IB) school in Abu Dhabi. While still relatively young, it is a few years older than School A. School B has been inspected using the new inspection framework which gives greater focus on innovation. The following comprises the part of the reports pertaining to "Development and promotion of innovation skills":
\begin{quote}
Students demonstrate excellent communication skills. They use these very effectively to share ideas and explain their thinking. They confidently ask questions and challenge each other as, for example, when Year 8 use role play very creatively and make presentations. Students collaborate extremely effectively. They can work effectively in pairs or small groups, listen attentively to each other, negotiate their responses in older year levels and make creative and thoughtful presentations. Children in KG are empowered in a well-structured environment to make choices and develop secure independent skills and creativity in a range of situations. Students in all years have acquired confident use of iPads and other digital technologies and use them in all subject areas.

Students have a very strong work ethic when working independently and within groups. This is highlighted in collaborative participation in the secondary school Island Project, responding to STEM (science learning network) challenges, and the ‘Mantle of the Expert’ in the primary school. These develop high levels of research, problem-solving and critical thinking.

Most teachers provide opportunities for students to problem solve and respond to probing questions in planning their lessons. Students routinely respond confidently to 'why', 'can you explain', 'how' questions. By Years 12 and 13, students confidently ask challenging questions of each other.
Curriculum review and modification leads to extensive opportunities for students in all years to innovate, enhance their learning and show enterprise. For example, using iPads within the continuous provision in FS and Years 1 and 2; the accelerated reader programme in primary and secondary; and Year 9 managing the complex process of publishing a high quality book of student writing.

Senior leaders have embedded a culture of innovation into curriculum planning and development. Staff embrace opportunities and appreciate the support and encouragement to promote innovation within teaching and learning.
\end{quote}

In the report summary, the following was highlighted:
\begin{quote}
Students acquire key skills to innovate and be creative. They show very high understanding of environmental sustainability through the Eco club, participating in projects such as recycling, cleaning the beach or the solar-powered car challenge.

[...] In the more effective lessons, teachers very effectively promote innovation, creativity, research and critical thinking skills. For example, use of digital technologies in all subjects and year levels to enhance research and recording. Year 9 students prepare ‘flipped’ learning videos to lead learning in art lessons; and students prepare and publish a compendium of students’ writing.

[...] Innovative approaches to involving students in the planning, delivery and evaluation of charity events and curriculum enrichment activities develops their enterprise, innovation and collaborative skills. All students are prepared very well for their next phase
\end{quote}

The school's overall ADEC rating was Outstanding, with the section pertaining to innovation gaining the same grade. As one of the better schools in the region, as well as one of the first to be evaluated using the new framework, the insight that the report gives into how ADEC will in evaluating innovation is very useful.

One of the first things apparent in the report is what they have included relating to innovation. The first part relating specifically to the innovation skills contains evidence that would appear without any other context, would be difficult to be considered innovative. Collaborating, presentations and questioning are good practice and could well be part of an innovative lesson or scheme of work, but cannot really be considered as evidence themselves of innovation. This over-inclusion is interesting for a few reasons: firstly, this is the first glimpse of what ADEC will be looking for, which does seem to be fairly broad and secondly the inspection team is not directly part of ADEC so this could perhaps explain the disconnect in values. They also highlighted some good use of technology which, as previously discussed, is a contentious issue when it comes to innovation, as well as a cross-curricular project which the school developed. TODO: has it been discussed?

The cross-curricular project was rightly highlighted by both the inspection and a member of staff interviewed as a good example of innovation. Students participated in a scenario-based project over 6 weeks which encompassed Art, English, DT, Music, History, Geography, Maths, Science, ICT, Languages, Arabic and Social Studies. As well as covering a wide-range of subjects, it also gave students opportunity to work on soft skills such as communication and teamwork. 

Whilst not a pioneering idea, most Primary-trained teachers would be able to show comparative ideas in their part of the school, but doing so in a Secondary school with the array of subjects involved and the time span of a half-term is an excellent example innovative curriculum design.

In terms of the type of innovation identified, curriculum design, teaching and learning and student innovation skills were all mentioned in the report. Innovation within the leadership was briefly mentioned but was not quantified or explained. From interviews, it is clear that School B diplays a clear ethos of innovation within management but perhaps an inspection does not have sufficient time to see this in sufficient depth. 

\subsubsection{School C}
TODO: Pamoja

School C is a premium school located in Dubai offering a mixture of British and IB curricula, the latter is delivered to post-16 students only. In its inspection report, the following was mentioned about innovation:

\begin{quote}
The school's mission and vision aligned closely with the national innovation agenda. The strategic plan provided clear direction for further development of this highly innovative school. Innovation was an intrinsic characteristic, promoted by all stakeholders, including innovation mentors and external partners.

The building design offered diverse work areas, including open learning plazas. The learning environments included ICT that allowed students choice in both what and how they learned. Extended time blocks enabled learning that was inquiry-based, focused on thinking skills, and connected to other subjects in purposeful ways. An extensive enrichment program provided students with variety, choices and challenge.
\end{quote}

Its main focus for innovation is focused around learning spaces, lesson delivery and curriculum changes to support the learning spaces. Two years ago, School C introduced a large learning space which was fitted with various different arrangements of furniture, aimed at providing maximum flexibility to staff as students as to how to use the space. The goal of the space was two-fold: to try to innovate with the physical space and to use that as inspiration to innovate in other areas that followed, such as delivery and curriculum design. 

Like many schools in the area, the school operates a bring your own device (BYOD) policy where students provide their own laptop for use at school. If you combine BYOD, wireless connectivity (a pre-requisite for BYOD) and a flexible learning space, there is no need for students to be sat at desks in rows in front of a teacher. This was the foundation upon which the innovation work was to be built.

The obvious and immediate impact the space had on lessons that chose to utilise it was a marked reduction in teacher-led lessons and a move towards student-led learning and knowledge discovery. With people rightly questioning (albeit somewhat tongue-in-cheek) the role of teachers alongside the vast array of information available on the Internet \cite{gilbert2010need}, is not only something very current in educational thinking, but student-led learning is also one of the ways ADEC have identified as being innovative.

The ambitious goal of the project is, at KS2/KS3, to move towards students having a greater degree in flexibility with the timing and duration of their learning. If knowledge is no longer derived from teachers directly and learning resources and assessments are available online, in theory, students could decide when and where their learning take place. Of course in practical terms, there are real issues to overcome, such as monitoring students' progress, ensuring students are engaged and focused, but these were deemed issues to be resolved as the project developed.

Any innovation which is developing new ideas does not always succeed in its goal, but there is a fine line between innovating and failure to learn from past mistakes. By all measures, the learning space had short-term success, whereas there have been other instances of it failing completely such as Bexhill High School in the UK \cite{Lusher2015}. Two important differences between the two cases are that School C did not replace classrooms with the new learning space and merely used it as an optional space in the first instance, reflecting a phased approach will be discussed later in this research. The second difference in its approach was the (externally-observed) quality of the schools; Bexhill was at the time, or shortly after in Special Measures, whereas School C was rated Good.\footnote{Equivalent to an Unacceptable judgement in Abu Dhabi} Innovation in schools who are not doing the fulfilling basic educational measures will inevitably lower the probability of success for anything innovative. Innovation cannot succeed without solid fundamentals and a confidence to take risks that comes with that.

School C placed a great emphasis on innovation: it formed a group of teachers to form a research group, primarily focused on the use of the flexible space, but anything was allowed to be proposed on the basis it was considered to be innovative. Proposals were solicited from SLT to the entire staff body for ideas, from which a selection was made. These teachers were then awarded a financial allowance to perform research into their chosen topic. 

Some of the points raised from teachers in their questionnaire responses were that schools and teachers needed to be more research-based, and more importantly, allowed the time to do so. Whilst School C did not allow time, a financial reward was given instead. For this school this was no doubt a reasonably-sized investment financially, but this research group idea was the most effective, productive and innovative idea observed. Looking at one of our earlier definitions of innovation, "planned deliberate change ... but it does not necessarily result in [enhancement]" \cite{hannan2002innovative}, a group where innovations are multiple and iterated upon can only result in a higher chance of good ideas emerging. We know that innovation does not always result in success, so limiting innovation to large-scale projects will inevitably end in a lower rate of success as failure of larger projects will not only tie up resources but could lead to a negative perception of the innovation project.

The model of this research, whether deliberately or not, mirrors the rapid prototyping model seen in multiple areas such as software development and manufacturing. There are also some studies that show this methodology can be quite effective at introducing innovative changes in a more low-key manner than would normally be seen in most schools: "[...] research on organizations also indicates that innovations assimilate into the culture of the environment more quickly if they quietly are homogenized into the corporate culture without mandates. Once an innovation [...] becomes less threatening, there is a kind of contagion; the use of the innovation spreads to more employees, reaches a tipping point of critical mass, and eventually becomes ubiquitous. It is then seamlessly incorporated into the process and practice of the participants." \cite{cook2008combining}

This contagion effect was observed wherein teachers who would otherwise themselves not have had time to innovate or perform action research, were noting the results from projects and adopting them as they became more mature. There was little investment from the participants so each idea stood on its own merit: if it improved lessons, curricula or assessment, it was quietly adopted. If the barriers to entry were too large or results were not seen, it was not.

The aim of moving towards a blended approach ("combining online and face-to-face instruction" \cite{bonk2006handbook}) is already underway with their sixth form IB curriculum already being delivered in a blended manner with a collaboration with Pamoja Education.\footnote{http://www.pamojaeducation.com/} Using a blended approach in this way offers several benefits: students can choose from a wider array of subjects as teachers do not have to be employed by the school (and are more likely to be IB-specialists) and students can work at their own pace and with a greater amount of independence. While a different approach to School A, it is similar in its goals and outcomes for students.

TODO: pad these out

\subsection{Good Practice}
Three schools were chosen to be investigated in more depth and have had observational case studies created detailing their innovations and projects relating to innovation. They were chosen because of their reputation for innovation, whether that was from inspection reports, personal experience, recommendations or otherwise. There were many other schools evaluated for inclusion in this research; while it cannot claim to be exhaustive or wholly inclusive, it aims to provide a snapshot of what some innovative schools are doing. 

In these innovative schools, there is a real sense that firstly students are not being adequately prepared for university, a sentiment shared by many, \citep{Ali2016,Moussly2012} and secondly that while the education standard in the UAE on average is not amongst the best in the word yet, \cite{2013} becoming a high quality educational institution is a very achievable goal, and also one that the UAE can has indeed set for itself \cite{UAEGovernment2012}.

\subsubsection{Solid Foundation}
The schools were all confident in their teaching and learning. Without getting the basics right, introducing innovation is less likely to succeed. The implication is that trying to innovate without a solid foundation is like running before you can walk. This is also expressed in inspection reports with similar sentiment. TODO: Ref this and discuss

\subsubsection{Forward-thinking Leadership}
All schools had a leadership team in place that encouraged innovation 

\subsubsection{Research-based model}
The innovation research group seen in School C, while not financially possible in all schools, 
showed the benefit of a first-class approach to innovation and research

\subsubsection{New Teaching Ideas}
Pushing the boundaries with teaching whether it be a blended approach, cross-curricular projects or the Harkness method, these will undoubtedly find new opportunities for both staff and students to enhance learning 

\subsubsection{Attitude}
All schools had an attitude that failure was acceptable when innovating, something critical to being a successful innovator 

\subsection{Opinions on Innovation}

The nature of the innovation drive, i.e. from the Government, will lead to an array of differing opinions.

\subsection{Luxury}
2/3 schools not \cite{ADEC2016 pi.3}

\subsection{Resources}
A common theme during this research process was the slight confusion and hesitation towards implementation at the very start of the process due to the lack of guidance initially, and the fact there had not been any feedback from inspections. Late in the research process a publication surfaced entitled "A Guide to Development and Promotion of Innovation Skills" by ADEC \citet{ADEC2015a}. This document does not appear online anywhere, and while is it referenced in an online newsletter, no digital version appears to exist online. Requests to ADEC for a soft copy went unanswered, so the guide is included as Appendix \ref{appendix:adec}.\footnote{https://www.adec.ac.ae/en/MediaCenter/Publications/Teaching\%20Matters\%20Newsletter-Issue\%2017/files/assets/common/downloads/publication.pdf} A straw poll survey of schools indicate that they did not receive a copy of this, but this could not be confirmed across all schools. This document addresses a lot of problems with how innovation is defined, framed and assessed.

\subsection{Other Themes}
* Polarising opinion on technology

\subsection{Future of Innovation}

The way innovation is set out in both educational documents as well as more general documents, it is clear that innovation is here to stay; it is a fundamental foundation for UAE’s society and education system for the foreseeable future.
This research has focused on the early stages of implementation for schools becoming more innovative, but what has that told us about how the future might look?

As a Computing teacher, the focus of innovation is something I have welcomed, as it is a focus which goes hand-in-hand with my subject. Outside of the classroom, if schools are being pushed to be more innovative, teachers such as myself will have the freedom and encouragement to be more innovative in their classroom. Teachers will be taken “along for the ride” as new ideas get tested, good ideas shared, and teaching practice improved.

I am fortunate to work at a school that even without this focus, it would be prioritising innovation in other ways. The focus gives the school a licence to push the boundaries even further than would be possible without it. Schools in the UAE who want to excel need to hire the best teachers from around the world. In order to do that, they need to be an attractive prospect. If a school gains a reputation for being an innovative school, teachers are more likely to take a chance on the school, and more importantly, stay. 

One of the common problems for UAE schools previously mentioned, is the transient nature of both staff and students. An increase in the profile of a school would tempt teachers to stay for longer, which would obviously benefit the school. On a larger scale, an improvement in some schools and teaching via the innovation drive would no doubt improve the overall quality of schools. Education in the UAE is developing rapidly but it is still behind many countries, especially when it comes to areas such as CPD and training, mainly due to the size and location in the world. An improvement in school development and teaching would lead to more practitioners available to share good practice with others in the area.

It seems premature to discuss a global stage in relation to innovation, but one has only to look at the speed of development in neighbouring Dubai as a tourist, financial and transport hub to see what happens when everything comes together at the right time.
The UAE has several differences which make success on a global scale more likely than might be expected. It is a federal, absolute, monarchy, with one of the seven federal monarchs elected and the President (historically, the Abu Dhabi ruler) and one as the Prime Minister (historically, the ruler of Dubai). Laws and initiatives can be enacted quickly without too much red tape and zero resistance which is why large visions such as Dubai’s rapid development can be achieved.

Until now, Abu Dhabi, Dubai and the northern Emirates have worked independently on their own education systems. While there is still a way to go, the integrated inspection framework is an indicator of the more joined-up thinking that is planned for the UAE education system. \textbf{TODO: any news refs?}

One of the slightly more negative future prospects surrounds the idea of inspecting innovation. If innovation is seen as a burden for some schools rather than a positive driving force, they will become savvy at showing inspectors what they want to see. Going against this is the weight given to it in the framework, but in the inspection report it is included under learning skills with five others. Upon reading existing reports, a pattern of what needs to be shown becomes clear which could provide schools an easy box to tick. The long-term solution to this seems to be maintaining emphasis and support on innovation so it does not become a short-term idea. 

\subsection{Student Competency Framework}

Late in the process of this research, ADEC introduced a phased implementation of a new, but closely related, project called the Student Competency Framework (SCF). As an ADEC project, it was also authored in part by Pearson, the British publishing and education company. Similar to many of this type of framework, which it makes reference to, it is a set of expected skills students are meant to master to become more rounded students or citizens. The SCF is comprised of three sets of skills, one of which is "Learning and innovation skills", defined as: "those skills which enable students to be become creative, innovative, flexible life-long learners who are able to be effective, multi-skilled productive workers". This is strong evidence that ADEC sees innovation as a multidiscipline idea, as they have added student innovation skills to the existing innovation areas examined in their inspections.

While it is hard to identify the specific origin of individual sentiments, it can be assumed that whoever the original author is, the other party would presumably agree given the joint publication. Unlike the introduction of innovation, this was designed as a phased implementation. The obvious reason for this would be the larger scale of the project. The justification for a phased implementation is given as:

\begin{quote}
Change management in education and the implementation of innovation require careful strategic planning. Too many good initiatives have failed through limiting their roll-out strategies to a dissemination of information (Cordingley and Bell, 2007). Dissemination alone will not succeed in embedding large scale innovation in the long term. Research makes clear that a clear, structured well-integrated strategy that ensures new initiatives are embedded and then sustained by all stakeholders is required for successful delivery and implementation. Securing ‘buy-in’ is critical. This is usually achieved by ensuring that stakeholders understand and commit to the required change. This is most likely when the change is seen to benefit learners, as is the case with the SCF.
\end{quote}

This is a fairly different approach than the the one taken when adding innovation itself as an inspection criteria: the roll-out was a dissemination of information (in this case a change to the inspection framework) and there were few strategies proposed which resulted in few stakeholders understanding how to be innovative. This difference will be further expanded on in a broader fashion in later chapter. \textbf{TODO: Where?}

Given the previous discussion of the definition and examples of innovation and given that the SCF will aim in part to teach students to be innovative, it will be useful to see what is written about innovation.

The first point to note is that innovation is paired with creativity to form a common theme. This reflects some of the earlier evidence found when trying to define innovation; it is very closely linked to creativity.

It is important to note when reading the SCF that innovation is both listed as a desired student skill, as well as a description for the implementation of the entire project. It is the latter where the framework has really excelled in providing research.

As shown in Figure \ref{fig:implementation}, a successful implementation of an innovative project requires all five successful elements of: vision, skills, incentives, resources and action plans. Working backwards from the end criteria of confusion, which we have previously identified, this model indicates that there is a lack of vision. Looking at the vision, it is defined as trying to creative an innovative education system, with not much else.

Given the preparation and planning that has gone into the SCF document, what ideas can we extract from this that would have improved the innovation introduction?

\begin{itemize}
\item Context - the SCF places their version in the global context by comparing and contrasting with existing student frameworks and building upon them
\item Timescale - from start to end, the process will take seven years, covering a pilot and monitoring phase
\item Stakeholders - extensive work with stakeholders before and during the process to identify problems and opportunities
\item Support - one the planned tasks for ensuring buy-in is: "developing appropriate guidance and support materials"
\end{itemize}

The SCF is a larger project than the innovation addition, but "too many good initiatives have failed through limiting their roll-out strategies to a dissemination of information (Cordingley and Bell, 2007)" \cite{ADEC2016}
