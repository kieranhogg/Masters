\section{Chapter 2 - A Critical Review of the Literature}
4,000 – 5,000 words
One of the first challenges when approaching a review of the literature is to be able to define the scope of the search. In order to be able to accurately discover appropriate material, we must attempt to define innovation within the context of education. Given the simplicity of the term and the complexity of the meaning and implications, it comes as no great surprise that there is not singularly accepted definition. The best attempt to define it using the range of related definitions found during the review is as follows:

\begin{quote}
The variety of ways in which the concept of innovation has been defined by researchers reflects the nature of the discipline \cite{Gopalakrishnan_1994}, the level at which innovation is conceptualised, and whether it is being conceived as a product or process (Amabile, 1988; Kanter, 1988). According to one of the earlier definitions, innovation is an idea, practice or material artifact perceived to be new by the relevant unit of adoption \cite{Allen_1975}. Later, Anderson and King (1993) conceptualised innovation as the emergence, import or imposition of new ideas which are pursued towards implementation, through interpersonal discussion and successive remoulding of the original proposal over time. This contemporary definition not only describes the nature of innovation, but also refers to the intrinsic process of implementation. With advances in research, the concept of innovation has also been refined and a more comprehensive understanding of innovation has emerged. It may be defined as the introduction and application within a group, organisation, or wider society, of processes, products or procedures new to the relevant unit of adoption and intended to benefit the group, individual or wider society (West & Farr, 1990).
\end{quote}  \cite{Sharma_2005}

This a more general definition than one would normally come across specifically in education, but it helps to give the full context of the derivation. The distinction of is being a product or a process \cite{} is one that will be discussed later.

A slightly more narrow definition, although one that is probably more fitting for this context is where innovation is differentiated from enhancement by the notion that innovation is "planned deliberate change ... but it does not necessarily result in [enhancement]." \cite{hannan2002innovative}

As the focus of the research is simultaneously broad (innovation is a large area of research) and narrow (specifically the implementation in the UAE) there are a wide range of topics that are generally related but very few that are close to this context. 

Areas that were studied during the literature review based around the following areas:
\begin{itemize}
\item Innovation Education - a Icelandic subject
\item Innovation within schools
\item Innovation within other establishments such as higher education
\ite Innovation as a concept
\end{itemize}

\subsection{Innovation Studies}
Innovation Studies is an Icelandic subject which appeared in 1999, and was introduced to their national curriculum from 2007. On first reading, a subject related to Design & Technology would not appear to hold much relevance to the broader idea of innovation aside from sharing the name. Reading the emphasis behind it however, it becomes clearer as to the similarities:
\begin{quote}
[Innovation Studies] is driven by an innovation process rather than subject content and as such is cross-curricula. Students capitalize on their
knowledge from all sources (Aòalnámskrá grunnskóla, 1999, 31). Ⅰn addition innovation exercises can provide a context for researching and further understanding.
\end{quote} \cite{thorsteinsson2005innovation}

This form of innovation is therefore as a subject; a set of skills which students are allowed to develop and practice in order to become innovative students. They had also identified a subject with which to build it in to, or to develop, that being Design & Technology.

This is an interesting facet of innovation given the other non-subject forms it are much more common, in practice and in the literature. I think this will be an interesting 

\subsection{Innovation Within Schools}
One of the more in-depth studies is a study focusing on a wide range of areas within Israeli schools. \cite{tubin2003domains} They devised a series of aspects on which they assessed each school, ultimately producing a score for the average level of innovation in each school. The schools were scored on:

\begin{itemize}
\item Physical space - use (or not) of the traditional classroom
\item Digital space - use of any technological "space" such as a VLE (Virtual Learning Environment)
\item Time - whether this was taking place within traditional lessons, virtual courses etc.
\item Student roles - developing the roles that students take within the classroom such as teaching assistants
\item Teacher roles (with students and teachers) - if teachers took on differing roles in the classroom and working with colleagues
\item Curriculum content - use of innovative techniques to enrich or change existing curricula
\item Didactic solutions - the movement away from traditional didactic approaches such as paper-based worksheets
\item Assessment methods - primarily the use of ICT to develop new or facilitate exiting assessment practices
\end{itemize}

The use of scoring to come up with a average innovation score is quite an interesting concept and the foci would seem to be a selection which most people could agree an innovative school would have some of them present. Where the study does reveal its age is the loose relationship of ICT equalling innovation. ICT is mentioned in the majority of areas looked at, with a comment on one school's innovative curriculum content stating that "the French language teacher asked her students to find their favorite French songs and food on the Internet to enrich their own Website". While a good example of a cross-curricular lesson, that would not pass for innovation anymore.
\cite{StopI4:online}