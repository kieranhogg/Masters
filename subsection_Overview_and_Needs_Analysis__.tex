\subsection{Overview and Needs Analysis}
\begin{quote}
\begin{itemize}

\item  What is the focus of your enquiry?
\item What questions are you trying to answer?
\item What are your hypotheses or hunches?
\item Why are these questions and hypotheses important to you and your institution?
\item What contextual information is important to bear in mind? (E.g. 'the school is single sex' or 'the children are year seven')

\end{itemize}
\end{quote}

\begin{quote}
In this section you need to:
•	Identify your professional, department and school needs
•	Link these to a national and or international context.

This section justifies your enquiry and why you are undertaking the development work. This section relates to School, National and international needs and policies. Therefore you may refer to School and department development plans, local, national and or international initiatives.
\end{quote}

Innovation is a word and concept most are familiar with, with it being an idea which is applicable to almost every aspect of every business, and it usually is. Aside from the bi-annual dip in July and December (when presumably most of the world is taking a break from work), the popularity of the term has remained fairly constant since 2004, albeit with a slight decrease.

As a word in the 21st century, for most people it would conjure images of technology and the Internet, indeed the most related phrase according to Google is "technology innovation"\footnote{https://www.google.com/trends/explore#q=innovation}. As an innovation, the increase in prevalence of home technology and access to the Internet is rightly the most common example. A close second is "business innovation" which is where the word is commonly used, indeed the companies that produce the aforementioned technology would be themselves good examples of innovation companies.

One of the areas perhaps not as traditionally thought of as being innovative is education, at least not to the layperson. The idea of an educator who holds knowledge and dictates that to students has changed somewhat over the last hundred years but the core idea of that is still seen in most classrooms even today. For those more familiar with educational developments will appreciate that while this is true, there have been, and continue to be many changes and innovations that take place within education. The three-part lesson has been introduced and remained popular; learning styles have come and are perhaps on the way out; personalisation and better understanding of SEN students has been a huge positive development for the better.

At any given time, there are numerous entities researching and implement different ideas and changes in education, some hugely innovative, some less so. Sometimes the impetus for change comes from Governments, sometimes from schools, sometimes from teachers and occasionally from other actors such as parents and students. This research will focus on a innovation drive introduced by the Government of Abu Dhabi in the United Arab Emirates.

The United Arab Emirates is one of the fastest developing countries in the world\footnote{It has been on the World Bank list of high-income economies since 1987 \cite{worldbank} and is 41st in the UN Human Development Index \cite{WorkforHumanDevelopment2015}}, having only been established in 1971, which is developing many of its public services at a rapid rate to match the massive influx of people migrating to the country. Education is no exception to this with an increase in both non-local educators working in education, and workers in other sectors bringing their children to be educated in the country. This has lead to an education system which is both rapidly catching up with that of much older countries, and one which does not have the history and baggage of those countries. 

The UAE is comprised of seven emirates, the largest and home to the eponymously-named capital city is Abu Dhabi. Schools in Abu Dhabi come in two forms: public schools which use the curriculum set by the Abu Dhabi Education Council (ADEC) and offer free education to Emirati children and private schools which are fee-paying and typically follow a foreign curriculum such as British or American. Whilst private schools are more independent in their operation and curriculum, ADEC still has an role in inspecting these schools to ensure overall quality and compliance across the emirate.

At the start of the 2015-2016 academic year, a unified inspection framework for schools in the UAE was published by the ADEC and their counterpart organisations from different emirates. One of the new additions to the framework, which otherwise looks broadly similar to most developed countries', was the addition of innovation as an additional focus. To put it into perspective, it is given roughly the same amount of space in the framework as more established educational ideas such as inclusion. ADEC discuss innovation as follows:

\begin{quote}
Innovation comes in many forms. There are innovations in the way schools are owned, organised and managed; in curriculum design models; in teaching and learning approaches, such as the ways in which learning technologies are used; classroom design including virtual spaces; assessment; timetabling; partnerships to promote effective learning and engagement in the economy; and the ways in which teachers and leaders are recruited, trained, developed and rewarded. These innovations can be small or large, recognisable or entirely new and different.

Innovation is driven by a commitment to excellence and continuous improvement. Innovation is based on curiosity, the willingness to take risks and to experiment to test assumptions. Innovation is based on questioning and challenging the status quo. It is also based on recognising opportunity and taking advantage of it. Being innovative is about looking beyond what we currently do well, identifying the great ideas of tomorrow and putting them into practice.
\end{quote} \cite[p.12]{ADEC2015}

The idea of innovation in education is not a new one but the inclusion in such a prominent position of an inspection framework indicates the importance that ADEC and the UAE is placing on innovation, indeed they make reference to the UAE Vision 2021 as justification to the inclusion of innovation as an educational focus. There are many references to innovation in the vision, but the clearest one states: "We want the UAE to transform its economy into a model where growth is driven by knowledge and innovation." \cite{UAEGovernment2012} The importance placed in this area, the slightly suprising inclusion and the lack of exposure to it were the driving factors for choosing this as an area to focus on.

Due to the age of the initiative, the research will be conducted in an inductive manner; that is there will be no formal hypothesis to test. This research will follow the development of this formalisation of innovation in the academic year following its introduction. The aim is to both gauge the response from schools to the addition, but also to collate best practices for others to use.


Hypothetically I think that there will have been some good work from schools so far but that how schools have interpreted it and implemented it so far will differ. That is not to say that this is a bad thing, one of the hopeful outcomes of this research is a clearer picture which will enable other educators to learn from others.

I think the difference in implementation will be a reflection of the diversity present in education in the UAE with various different types of educational establishments, staffed by many different nationalities, which are all at different stages of their educational lives.

Broadly speaking, the initiative will be looked at in the following ways: 

\begin{itemize}
\item What is innovation in education and what is the goal of the initiative?
\item What form of innovation has taken place in schools so far?
\item How consistent is the interpretation of innovation between schools across the region?
\item How do the above answers compare to a more developed educational system?
\item Has there been any impact so far? What is the predicted future impact?

\end{itemize}
